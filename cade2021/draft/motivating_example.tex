% !TeX root = ./paper.tex

\begin{figure}
\footnotesize
\begin{minipage}[t]{0.4\textwidth}
$$\begin{aligned}&\forall y.\add(0,y):=y\\
&\forall x,y.\add(\suc(x),y):=\suc(\add(x,y))\\
\end{aligned}$$
\hrule
$$\begin{aligned}&\even(0):=\top\\
&\even.2:\even(\suc(0)):=\bot\\
&\forall z.\even(\suc(\suc(z))):=\even(z)\\
\end{aligned}$$
\hrule
$$\begin{aligned}&\half(0):=0\\
&\half(\suc(0)):=0\\
&\forall x.\half(\suc(\suc(x)):=\suc(\half(x))\\
\end{aligned}$$\end{minipage}
\begin{minipage}[t]{0.5\textwidth}
$$\begin{aligned}&\size(\bnil):=0\\
&\forall x,y,z.\size(\bnode(x,y,z)):=\suc(\add(\size(x),\size(z)))\\
\end{aligned}$$
\hrule
$$\begin{aligned}
&\brotate(\bnil):=\nil\\
&\forall x,y,z,u,v.\brotate(\bnode(\bnode(x,y,z),u,v)):=\\&\quad\brotate(\bnode(x,y,\bnode(z,u,v)))
\end{aligned}$$
\end{minipage}
%$$\begin{aligned}&\forall y.\add(0,y)&&:=y&&\forall x.\nil\append x&&:=x\\
%&\forall x,y.\add(\suc(x),y)&&:=\suc(\add(x,y))&&\forall x,y,z.\cons(x,y)\append z&&:=\cons(x,y\append z)\\
%\cline{1-8}
%&\even(0)&&:=\top&&\fltn(\bnil)&&:=\nil\\
%&\even(\suc(0))&&:=\bot&&\forall u,v,w.\fltn(\bnode(u,v,w))&&:=\\
%&\forall z.\even(\suc(\suc(z)))&&:=\even(z)&&\phantom{aa}\fltn(u)\append\cons(v,\fltn(w))\\
%\cline{1-8}
%&\forall x.\fleq(0,x)&&:=\top&&\forall x.\fltn_2(\bnil,x)&&:=x\\
%&\forall x.\fleq(\suc(x),0)&&:=\bot&&\forall u,v,w,y.\fltn_2(\bnode(u,v,w),y)&&:=\\
%&\forall x,y.\fleq(\suc(x), \suc(y))&&:=\fleq(x, y)&&\phantom{aa}\fltn_2(u,\cons(v,\fltn_2(w,y)))\\
%\end{aligned}$$
\caption{Recursive function definitions used}
\label{fig:functions}
\end{figure}
\normalsize

Finding an induction formula that leads to a successful proof for an inductive problem is known to be undecidable. Some provers like \textsc{ACL2} or \textsc{IsaPlanner} take into account the structure of any function present in an inductive goal when creating induction formulas and provide heuristics to select the most suitable one as the next proof step. Such heuristics are however still relatively uncommon in saturation-based theorem provers.
\begin{example}\label{ex:1}
$$\forall x,y. (\even(x)\land \even(y))\rightarrow \half(\add(x,y))=\add(\half(x),\half(y))$$
A manual proof starts with selecting a suitable induction formula and proving its antecedent. Denoting the formula by $\forall x.F[x]$, one such formula is:
$$\begin{pmatrix}F[0]\land F[\suc(0)]\land\forall z.(F[z]\rightarrow F[\suc(\suc(z))])\end{pmatrix}\rightarrow\forall x.F[x]$$
Now to prove the antecedent, we prove each of its conjuncts:
\begin{enumerate}
	\item[(1)] $F[0]$: By simplifying with axioms of $\even$, $\half$ and $\add$ (shown in Figure \ref{fig:functions}), we get $\even(y)\rightarrow \half(y)=\half(y)$, a tautology.
	\item[(2)] $F[\suc(0)]$: By recognizing that $\even(\suc(0))$ is false, the implication is true.
	\item[(3)] $F[\suc(\suc(z))]$ given $F[z]$:
	This can be simplified to get $(\even(z)\land \even(y))\rightarrow \suc(\half(\add(z),y))=\suc(\add(\half(z),\half(y)))$. By the term algebra injectivity axiom, we get exactly the induction hypothesis.
\end{enumerate}
\end{example}
The task of selecting proper induction formulas gets harder as we have more functions and more complicated term structure. Even more so when it comes to automation -- most state-of-the-art automated theorem provers fail to select the right induction formula here. Let us look at a different example.
\begin{example}\label{ex:2}
	The following formula conjectures that the number of non-leaf nodes of a binary tree does not change after rotating it clockwise as many times as possible in its root:
	$$\forall x.\size(\brotate(x))=\size(x)$$
	We now proceed to create an induction formula. For now, let us focus on the step case: we can easily see that a term $\bnode(u,v,w)$ does not match any axiom of $\brotate$, so we better off using e.g. $\bnode(\bnode(x,y,z),u,v)$:
	$$\size(\brotate(\bnode(\bnode(x,y,z),u,v)))=\size(\bnode(\bnode(x,y,z),u,v))$$
	Simplifying this yields the following:
	$$\size(\brotate(\bnode(x,y,\bnode(z,u,v))))=\suc(\suc(\add(\add(\size(x),\size(z)),\size(v))))$$
	Usually, induction hypotheses with terms $\bnode(x,y,z)$, $v$, $x$, or $z$ would be used to rewrite the right-hand side but none of them help get rid of $\bnode(x,y,\bnode(z,u,v))$ on the left-hand side. One simple solution is to use the non-trivial induction hypothesis stemming from the third axiom of $\brotate$:
	$$\size(\brotate(\bnode(x,y,\bnode(z,u,v))))=\size(\bnode(x,y,\bnode(z,u,v)))$$
	The order underlying this is well-founded since no finite binary tree can be rotated indefinitely in one direction. After simplification of the hypothesis, we get:
	$$\size(\brotate(\bnode(x,y,\bnode(z,u,v))))=\suc(\add(\size(x),\suc(\add(\size(z),\size(v)))))$$
	Rewriting the step case with the hypothesis and invoking term algebra injectivity gives:
	$$\suc(\add(\add(\size(x),\size(z)),\size(v)))=\add(\size(x),\suc(\add(\size(z),\size(v))))$$
	This can be generalized to the relatively easily provable theorem:
	$$\forall x,y,z.\suc(\add(\add(x,y),z))=\add(x,\suc(\add(y,z)))$$
\end{example}

In this paper, we create "matching" induction formulas using function definitions -- this gives us the necessary case distinction and hypotheses for each function -- and if different induction formulas are generated this way, combine these such that the case distinctions and hypotheses are preserved.

In a saturation-based theorem prover however, even the "correct" induction formula is useless if the inference rules have limitations, preventing us from eventually solving each case. One such issue comes up when considering our current induction inference using only one premise (Section \ref{sec:preliminaries}) with which we cannot solve the second case of Example \ref{ex:1} since it does not provide us with the false condition $\even(\suc(0))$ necessary for that case.

Another issue is illustrated with the step case of Example \ref{ex:2}. Term orderings usually do not allow using the second axiom of $\size$ in the intended left-to-right orientation because $\suc(\add(\size(x),\size(z)))$ is larger in a lot of term orderings than $\size(\bnode(x,y,z))$. We address this issue by modifying the superposition calculus such that function definition axioms are used as rewrite rules.
