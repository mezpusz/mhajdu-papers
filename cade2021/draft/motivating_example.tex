% !TeX root = ./paper.tex

\begin{figure}
\footnotesize
%\begin{minipage}[t]{0.4\textwidth}
%$$\begin{aligned}&\forall y.0+y:=y\\
%&\forall x,y.\suc(x)+y:=\suc(x+y)\\
%\end{aligned}$$
%\hrule
%$$\begin{aligned}&\even(0):=\top\\
%&\even.2:\even(\suc(0)):=\bot\\
%&\forall z.\even(\suc(\suc(z))):=\even(z)\\
%\end{aligned}$$
%\hrule
%$$\begin{aligned}&\forall x.0\leq x:=\top\\
%&\forall x.\suc(x)\leq 0:=\bot\\
%&\forall x,y.\suc(x)\leq \suc(y):=x\leq y\\
%\end{aligned}$$\end{minipage}
%\begin{minipage}[t]{0.5\textwidth}
%$$\begin{aligned}&\forall x.\nil\append x:=x\\
%&\forall x,y,z.\cons(x,y)\append z:=\cons(x,y\append z)\\
%\end{aligned}$$
%\hrule
%$$\begin{aligned}
%&\fltn(\bnil):=\nil\\
%&\forall u,v,w.\fltn(\bnode(u,v,w)):=\\&\quad\fltn(u)\append\cons(v,\fltn(w))
%\end{aligned}$$
%\hrule
%$$\begin{aligned}
%&\forall x.\fltn_2(\bnil,x):=x\\
%&\forall u,v,w,y.\fltn_2(\bnode(u,v,w),y):=\\&\quad\fltn_2(u,\cons(v,\fltn_2(w,y)))$$
%\end{minipage}
$$\begin{aligned}&\forall y.0+y&&:=y&&\forall x.\nil\append x&&:=x\\
&\forall x,y.\suc(x)+y&&:=\suc(x+y)&&\forall x,y,z.\cons(x,y)\append z&&:=\cons(x,y\append z)\\
\cline{1-8}
&\even(0)&&:=\top&&\fltn(\bnil)&&:=\nil\\
&\even(\suc(0))&&:=\bot&&\forall u,v,w.\fltn(\bnode(u,v,w))&&:=\\
&\forall z.\even(\suc(\suc(z)))&&:=\even(z)&&\phantom{aa}\fltn(u)\append\cons(v,\fltn(w))\\
\cline{1-8}
&\forall x.0\leq x&&:=\top&&\forall x.\fltn_2(\bnil,x)&&:=x\\
&\forall x.\suc(x)\leq 0&&:=\bot&&\forall u,v,w,y.\fltn_2(\bnode(u,v,w),y)&&:=\\
&\forall x,y.\suc(x)\leq \suc(y)&&:=x\leq y&&\phantom{aa}\fltn_2(u,\cons(v,\fltn_2(w,y)))\\
\end{aligned}$$
\caption{Recursive function definitions used}
\label{fig:functions}
\end{figure}
\normalsize

Finding an induction formula that leads to a successful proof for an inductive problem is known to be undecidable. Some provers like \textsc{ACL2} or \textsc{IsaPlanner} take into account the structure of any function present in an inductive goal when creating induction formulas and provide heuristics to select the most suitable one as the next proof step. Such heuristics are however still relatively uncommon in saturation-based theorem provers.
\begin{example}\label{ex:1}
$$\forall x,y. (\even(x)\land \even(y))\rightarrow \even(x+y)$$
The function definitions for $\even$ and + can be found in Figure \ref{fig:functions}. We can do the straightforward derivation:
\footnotesize
\begin{mdframed}[usetwoside=false,innertopmargin=-5pt,skipabove=0pt,skipbelow=0pt]\begin{equation}\nonumber\begin{aligned}
&\even(\sigma_0)&&\text{1. input}\\
&\even(\sigma_1)&&\text{2. input}\\
&\neg\even(\sigma_0+\sigma_1)&&\text{3. input}\\
\hline
&\begin{pmatrix}\even(0+\sigma_1)\land\\\forall z.(\even(z+\sigma_1)\rightarrow\even(\suc(z)+\sigma_1))\end{pmatrix}\rightarrow \forall x.\even(x+\sigma_1)&&\text{4. induction f.}\\
&\neg\even(0+\sigma_1)\lor \even(\sigma_2+\sigma_1)&&\text{5. bin.res. 3, cnf(4)}\\
&\neg\even(0+\sigma_1)\lor \neg\even(\suc(\sigma_2)+\sigma_1)&&\text{6. bin.res. 3, cnf(4)}\\
&\neg\even(\sigma_1)\lor \even(\sigma_2+\sigma_1)&&\text{7. 5, + axiom}\\
&\neg\even(\sigma_1)\lor \neg\even(\suc(\sigma_2)+\sigma_1)&&\text{8. 6, + axiom}\\
&\even(\sigma_2+\sigma_1)&&\text{9. bin.res. 2, 7}\\
&\neg\even(\suc(\sigma_2)+\sigma_1)&&\text{10. bin.res. 2, 8}\\
&\neg\even(\suc(\sigma_2+\sigma_1))&&\text{11. 10, + axiom}\\
\hline
&...\\
&\even(\suc(\sigma_3+\sigma_1))&&\text{12.}\\
&\neg\even(\sigma_3+\sigma_1)&&\text{13.}
\end{aligned}\end{equation}\end{mdframed}
\normalsize
After selecting (3) for induction and binary resolving it with the clausal form of the simplest induction formula (4) with $\sigma_0$ as induction term, we get (5) and (6). Simplifications give (9) and (11) but there is no $\even$ axiom to further simplify. One more induction on $\sigma_2$ and similar simplifications yield (12) and (13). Neither of the hypotheses (9) and (12) match the conclusions (11) and (13), either due to the different Skolem constants or the different term structure.

Using $\sigma_1$ as induction term does not help either -- we cannot get rid of any constructor terms in the second argument position of +.
\end{example}
In this paper, we create "matching" induction formulas using function definitions and then -- if different induction formulas are generated for a goal -- combine these in a way that all function terms can be simplified in the case distinction while keeping the necessary hypotheses.

In a saturation-based theorem prover, even the "correct" induction formula is useless if the term ordering prohibits rewriting function terms into their definitions or using necessary induction hypothesis, ultimately leading to a stuck proof.
\begin{example}
	Given a unit-clause:
	$$\{\fltn(\sigma_0)\append\sigma_1\neq\fltn_2(\sigma_0,\sigma_1)\}$$
	Proving this literal inductively with the simplest case distinction gives a step case $\bnode(u,v,w)$ with induction hypotheses $u$ and $v$. Due to the large terms on the right-hand side of function definitions $\fltn$ and $\fltn_2$ for case $\bnode(u,v,w)$, the induction step conclusion cannot be simplified:
	$$\fltn(\bnode(\sigma_2,\sigma_3,\sigma_4))\append\sigma_1\neq\fltn_2(\bnode(\sigma_2,\sigma_3,\sigma_4),\sigma_1)$$
\end{example}
So even though the right induction formula was found we could not solve one of its cases because of limitations in the calculus. We address this issue in a way that still preserves relative completeness and some of the properties of simplifications.