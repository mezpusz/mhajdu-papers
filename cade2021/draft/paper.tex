% This is samplepaper.tex, a sample chapter demonstrating the
% LLNCS macro package for Springer Computer Science proceedings;
% Version 2.20 of 2017/10/04
%
\documentclass[runningheads]{llncs}
%
\usepackage{amsmath}
\usepackage{mathtools}
\usepackage{amssymb}
\usepackage{rotating}
\usepackage{array}
\usepackage{vampire_induction}
\usepackage{graphicx}
\usepackage[table]{xcolor}
\usepackage{multirow}
\usepackage{xspace}
\usepackage{xargs}
\usepackage{adjustbox}
\usepackage{proof}
\usepackage{cancel}
\usepackage{hyperref}
\usepackage{tabularx}
\usepackage{mdframed}
\usepackage{hhline}

{
\small 
\newcolumntype{R}[2]{%
	>{\adjustbox{angle=#1,lap=\width-(#2)}\bgroup}%
	l%
	<{\egroup}%
}
\newcommand*\rot[1]{\multicolumn{1}{R{#1}{1em}}}


\makeatletter
%\renewcommand\section{\@startsection{section}{1}{\z@}%
%                       {-8\p@ \@plus -4\p@ \@minus -4\p@}%
 %                      {8\p@ \@plus 4\p@ \@minus 4\p@}%
 %                      {\normalfont\large\bfseries\boldmath
 %                       \rightskip=\z@ \@plus 8em\pretolerance=10000 }}
\makeatother
\newcommand{\AV}[1]{{\color{red}{AV: #1}}}

\newcommand{\todo}[1]{\emph{\color{red}{TODO: #1}}}

% Used for displaying a sample figure. If possible, figure files should
% be included in EPS format.
%
% If you use the hyperref package, please uncomment the following line
% to display URLs in blue roman font according to Springer's eBook style:
% \renewcommand\UrlFont{\color{blue}\rmfamily}

\begin{document}
%
\title{Structural Induction in Superposition Reasoning}
%
\titlerunning{ }
% If the paper title is too long for the running head, you can set
% an abbreviated paper title here
%
\author{M\'arton Hajd\'{u}\inst{1} \and Petra Hozzov\'a\inst{1} \and
  Laura Kov\'acs\inst{1\and 2} \and Andrei Voronkov\inst{3,}\inst{4}}
%
\authorrunning{ }
% First names are abbreviated in the running head.
% If there are more than two authors, 'et al.' is used.
%
\institute{
  TU Wien, Austria \and  Chalmers University of Technology, Sweden
  \and
  University of Manchester, UK
  \and
  EasyChair
}

%

\maketitle              % typeset the header of the contribution
%
%\vspace*{-1.3em}
\begin{abstract}
%\keywords{First keyword  \and Second keyword \and Another keyword.}

Formally ensuring software reliability is challenging due to the growing complexity of software in terms of its size, used unbounded and inductive data types and supported workflow. As manual code analysis is not a viable solution, automated reasoning approaches are needed for proving software correct. There are several prominent methods addressing this challenge by (i) translating parts of a system to a logic-based formalism and (ii) using automated theorem proving to validate certain safety and liveness program properties.

In this context a vital ingredient in automated reasoning is inductive reasoning, which can/must be used whenever repetition is present in a software component, e.g. loops or recursive data structures in computer programs or feedback loops in electric circuits. The two main branches in the literature
are explicit and implicit induction. In this thesis focus on the former approach which is based broadly on the Noetherian well-founded induction principle. There is a plethora of research already done in this area with several automated theorem provers as well as proof assistants are equipped with inductive reasoning techniques. Some of the earliest ones are \textsc{Nqthm/ACL2}, \textsc{Inka} and \textsc{Oyster/CLAM}. Recently, modern automated theorem provers, such as \textsc{ZipperPosition}, \textsc{CVC4} and \textsc{Vampire}, started to incorporate inductive types and reasoning.

The aim of the proposed thesis is to advance the state-of-the art in inductive reasoning within first-order theorem proving, by using automating reasoning with recursive function definitions over inductive data types.
\end{abstract}

\section{Introduction}
% !TeX root = ./paper.tex

This paper aims to contribute to the automation of proving problems with inductive types in the area of saturation-based theorem proving. Specifically, it introduces (1) new techniques to generate induction formulas and (2) new inference rules in saturation-based proof search to tackle some of the issues inherent in this topic.

While induction itself has been a well-known mathematical tool for centuries, its automation in inductive proofs started only in the 70s with a sequence of theorem provers like ACL2, CLAM/OYSTER or IsaPlanner just to name a few. In the nowadays more and more prominent field of saturation-based theorem proving, induction has been relatively out of focus with some work done in e.g. ZipperPosition, Vampire or CVC4. These initial inductive techniques mostly utilize basic induction formulas for term algebra datatypes and integers and during saturation very little help is provided for the prover to be able to actually solve the problems. We address several of these issues in this paper.



\section{Motivating Examples}
\label{sec:motivating}
% !TeX root = ./paper.tex

\begin{figure}
\footnotesize
\begin{minipage}[t]{0.4\textwidth}
$$\begin{aligned}&\forall y.\add(0,y):=y\\
&\forall x,y.\add(\suc(x),y):=\suc(\add(x,y))\\
\end{aligned}$$
\hrule
$$\begin{aligned}&\even(0):=\top\\
&\even.2:\even(\suc(0)):=\bot\\
&\forall z.\even(\suc(\suc(z))):=\even(z)\\
\end{aligned}$$
\hrule
$$\begin{aligned}&\half(0):=0\\
&\half(\suc(0)):=0\\
&\forall x.\half(\suc(\suc(x)):=\suc(\half(x))\\
\end{aligned}$$\end{minipage}
\begin{minipage}[t]{0.5\textwidth}
$$\begin{aligned}&\size(\bnil):=0\\
&\forall x,y,z.\size(\bnode(x,y,z)):=\suc(\add(\size(x),\size(z)))\\
\end{aligned}$$
\hrule
$$\begin{aligned}
&\brotate(\bnil):=\nil\\
&\forall x,y,z,u,v.\brotate(\bnode(\bnode(x,y,z),u,v)):=\\&\quad\brotate(\bnode(x,y,\bnode(z,u,v)))
\end{aligned}$$
\end{minipage}
%$$\begin{aligned}&\forall y.\add(0,y)&&:=y&&\forall x.\nil\append x&&:=x\\
%&\forall x,y.\add(\suc(x),y)&&:=\suc(\add(x,y))&&\forall x,y,z.\cons(x,y)\append z&&:=\cons(x,y\append z)\\
%\cline{1-8}
%&\even(0)&&:=\top&&\fltn(\bnil)&&:=\nil\\
%&\even(\suc(0))&&:=\bot&&\forall u,v,w.\fltn(\bnode(u,v,w))&&:=\\
%&\forall z.\even(\suc(\suc(z)))&&:=\even(z)&&\phantom{aa}\fltn(u)\append\cons(v,\fltn(w))\\
%\cline{1-8}
%&\forall x.\fleq(0,x)&&:=\top&&\forall x.\fltn_2(\bnil,x)&&:=x\\
%&\forall x.\fleq(\suc(x),0)&&:=\bot&&\forall u,v,w,y.\fltn_2(\bnode(u,v,w),y)&&:=\\
%&\forall x,y.\fleq(\suc(x), \suc(y))&&:=\fleq(x, y)&&\phantom{aa}\fltn_2(u,\cons(v,\fltn_2(w,y)))\\
%\end{aligned}$$
\caption{Recursive function definitions used}
\label{fig:functions}
\end{figure}
\normalsize

Finding an induction formula that leads to a successful proof for an inductive problem is known to be undecidable. Some provers like \textsc{ACL2} or \textsc{IsaPlanner} take into account the structure of any function present in an inductive goal when creating induction formulas and provide heuristics to select the most suitable one as the next proof step. Such heuristics are however still relatively uncommon in saturation-based theorem provers.
\begin{example}\label{ex:1}
$$\forall x,y. (\even(x)\land \even(y))\rightarrow \half(\add(x,y))=\add(\half(x),\half(y))$$
A manual proof starts with selecting a suitable induction formula and proving its antecedent. Denoting the formula by $\forall x.F[x]$, one such formula is:
$$\begin{pmatrix}F[0]\land F[\suc(0)]\land\forall z.(F[z]\rightarrow F[\suc(\suc(z))])\end{pmatrix}\rightarrow\forall x.F[x]$$
Now to prove the antecedent, we prove each of its conjuncts:
\begin{enumerate}
	\item[(1)] $F[0]$: By simplifying with axioms of $\even$, $\half$ and $\add$ (shown in Figure \ref{fig:functions}), we get $\even(y)\rightarrow \half(y)=\half(y)$, a tautology.
	\item[(2)] $F[\suc(0)]$: By recognizing that $\even(\suc(0))$ is false, the implication is true.
	\item[(3)] $F[\suc(\suc(z))]$ given $F[z]$:
	This can be simplified to get $(\even(z)\land \even(y))\rightarrow \suc(\half(\add(z),y))=\suc(\add(\half(z),\half(y)))$. By the term algebra injectivity axiom, we get exactly the induction hypothesis.
\end{enumerate}
\end{example}
The task of selecting proper induction formulas gets harder as we have more functions and more complicated term structure. Even more so when it comes to automation -- most state-of-the-art automated theorem provers fail to select the right induction formula here. Let us look at a different example.
\begin{example}\label{ex:2}
	The following formula conjectures that the number of non-leaf nodes of a binary tree does not change after rotating it clockwise as many times as possible in its root:
	$$\forall x.\size(\brotate(x))=\size(x)$$
	We now proceed to create an induction formula. For now, let us focus on the step case: we can easily see that a term $\bnode(u,v,w)$ does not match any axiom of $\brotate$, so we better off using e.g. $\bnode(\bnode(x,y,z),u,v)$:
	$$\size(\brotate(\bnode(\bnode(x,y,z),u,v)))=\size(\bnode(\bnode(x,y,z),u,v))$$
	Simplifying this yields the following:
	$$\size(\brotate(\bnode(x,y,\bnode(z,u,v))))=\suc(\suc(\add(\add(\size(x),\size(z)),\size(v))))$$
	Usually, induction hypotheses with terms $\bnode(x,y,z)$, $v$, $x$, or $z$ would be used to rewrite the right-hand side but none of them help get rid of $\bnode(x,y,\bnode(z,u,v))$ on the left-hand side. One simple solution is to use the non-trivial induction hypothesis stemming from the third axiom of $\brotate$:
	$$\size(\brotate(\bnode(x,y,\bnode(z,u,v))))=\size(\bnode(x,y,\bnode(z,u,v)))$$
	The order underlying this is well-founded since no finite binary tree can be rotated indefinitely in one direction. After simplification of the hypothesis, we get:
	$$\size(\brotate(\bnode(x,y,\bnode(z,u,v))))=\suc(\add(\size(x),\suc(\add(\size(z),\size(v)))))$$
	Rewriting the step case with the hypothesis and invoking term algebra injectivity gives:
	$$\suc(\add(\add(\size(x),\size(z)),\size(v)))=\add(\size(x),\suc(\add(\size(z),\size(v))))$$
	This can be generalized to the relatively easily provable theorem:
	$$\forall x,y,z.\suc(\add(\add(x,y),z))=\add(x,\suc(\add(y,z)))$$
\end{example}

In this paper, we create "matching" induction formulas using function definitions -- this gives us the necessary case distinction and hypotheses for each function -- and if different induction formulas are generated this way, combine these such that the case distinctions and hypotheses are preserved.

In a saturation-based theorem prover however, even the "correct" induction formula is useless if the inference rules have limitations, preventing us from eventually solving each case. One such issue comes up when considering our current induction inference using only one premise (Section \ref{sec:preliminaries}) with which we cannot solve the second case of Example \ref{ex:1} since it does not provide us with the false condition $\even(\suc(0))$ necessary for that case.

Another issue is illustrated with the step case of Example \ref{ex:2}. Term orderings usually do not allow using the second axiom of $\size$ in the intended left-to-right orientation because $\suc(\add(\size(x),\size(z)))$ is larger in a lot of term orderings than $\size(\bnode(x,y,z))$. We address this issue by modifying the superposition calculus such that function definition axioms are used as rewrite rules.


\section{Preliminaries}
% !TeX root = ./paper.tex

We assume familiarity with \textit{standard multi-sorted first-order logic with equality}. Functions are denoted with $f$, $g$, $h$, predicates with $p$, $q$, $r$ and variables with $x$, $y$, $z$, $u$, $v$, $w$, possibly with indices. We reserve the notation $\sigma$, $\sigma_0$, $\sigma_1$, etc. for Skolem constants. A term is \textit{ground} if it contains no variables. The notation $\overline{x}$ and $\overline{t}$ means tuples of variables and terms, respectively.

We use the standard logical connectives $\neg$, $\lor$, $\land$, $\rightarrow$ and $\leftrightarrow$ and quantifiers $\forall$ and $\exists$. \textit{Atoms} are built inductively from terms and predicate symbols. Atoms and their negations are called \textit{literals}. For a literal $l$, we use the notation $\overline{l}$ to denote its opposite sign literal. \textit{Formulas} are built from connectives and atoms.

Additionally, a disjunction of literals is a \textit{clause}. We reserve the symbol $\square$ for the \textit{empty clause} which is logically equivalent to $\bot$. We call every term, literal, clause or formula an \textit{expression}. We use the notation $s\trianglelefteq t$ to denote that $s$ is a \textit{subterm} of $t$ and $s\triangleleft t$ if $s$ is a \textit{proper subterm} of $t$.

We may use the words \textit{sort} and \textit{type} interchangeably. We distinguish special sorts called \textit{inductive sorts}, function symbols for inductive sorts called \textit{constructors} and \textit{destructors}. We require that the signature contains at least one constant constructor symbol for every inductive type. Such a symbol is called a \textit{base constructor}, while non-constant ones are called \textit{recursive constructors}. We call the ground terms built from the constructor symbols of a sort its \textit{term algebra}. Semantically, each $n$-ary constructor $c$ has $n$ corresponding destructors $d_1,...,d_n$. For any constructor term $c(t_1,...,t_n)$ with root symbol $c$, the following holds:
$$\forall 1\le i\le n. d_i(c(t_1,...,t_n))=t_i$$
Moreover, we usually axiomatise every term algebra with the \textit{injectivity}, \textit{distinctness}, \textit{exhaustiveness} and \textit{acyclicity} axioms. The inductive types we use in this paper are:
$$\begin{aligned}\nat&:=0 &&\mid \suc(\pre(\nat))\\
\lst&:=\nil&&\mid\cons(\head(\nat),\tail(\lst))\\
\btree&:=\bnil &&\mid \bnode(\bleft(\btree),\bval(\nat),\bright(\btree))\end{aligned}$$

An \textit{interpreted symbol} is a function or predicate whose meaning is defined through axioms, e.g. $=$ is an interpreted symbol in first-order logic with equality. All other symbols are called \textit{uninterpreted}. We distinguish \textit{function/predicate definitions} from regular axioms. These define a branch of computation for a function/predicate. Such axiom is denoted by marking exactly one equality literal in it with $:=$ such as $F\rightarrow l:=r$ which means that the orientation of this equality is fixed as left-to-right, $l$ is a function header and $r$ is a function definition, $F$ is the guard condition for this branch. We abuse this notation for predicate definitions where $:=$ can be replaced with a $\leftrightarrow$.

A notation we use is $E[s]$ meaning there is one (or more) distinguished occurrence(s) of the term $s$ in $E$. $E[t]$ then means that these occurrences are changed to a term $t$. 

A relation $R$ on a set $A$ is a \textit{simplification ordering} if:
\begin{itemize}
	\item it is \textit{stable under substitutions}\index{stable relation under substitutions}, i.e. $a\ R\ b$ implies $a\theta\ R\ b\theta$ for all $a, b\in A$ and substitutions $\theta$
	\item it is \textit{monotonic}\index{monotonic relation}, i.e. $a\ R\ b$ implies $s[a]\ R\ s[b]$ for all $a,b,s\in A$
	\item it has the \textit{subterm property}, i.e. $a\ \triangleleft\ b$ implies $a\ R\ b$
\end{itemize}
\subsection{Saturation-based proof search}
Given a set of input formulas $C$ in clausal form, the set of all derivable clauses using an inference system from the set $C$ is called the \textit{closure of $F$} w.r.t. the system. If the closure contains $\square$, the original set $C$ is unsatisfiable, otherwise it is satisfiable. The process of computing the closure is called \textit{saturation}. In practice more subtle notions are needed to tackle this problem, the first one is \textit{saturation up to redundancy}. A clause $C$ is \textit{redundant} w.r.t a set of clauses $S$ if some subset of $S$ of clauses smaller than $C$ w.r.t $\succ$ logically imply $C$. An inference system is usually equipped with \textit{simplification and deletion rules} to get rid of redundant clauses. Second, selection methods are used which control the order in which the inferences are applied. For a more detailed discussion on saturation algorithms see \cite{cav13}.

We use the \textit{superposition calculus} as the inference system in this paper. It works on sets of clauses -- we treat the conversion to \textit{clausal normal form} (or CNF) as a black box \cite{vcnf} and denote the conversion of a formula $F$ to its conjunctive normal form with $\mathtt{cnf}(F)$. The superposition calculus is \textit{sound} and \textit{refutationally complete}. A \textit{refutation} is a derivation of $\bot$. Refutational completeness means for any unsatisfiable formula set, we can derive the empty clause. Therefore, with superposition we usually negate our input conjecture and try to refute it which, if successful, means the original conjecture is valid.

For the completeness a simplification term ordering $\succ$ of terms is needed (e.g. KBO, LPO) which is extended to literals and clauses with the multiset-extension. We will abuse notation and use the same symbol $\succ$ to denote the original ordering and its extensions.

\section{Function definitions and Structural Induction}
\label{sec:fn_defs}
% !TeX root = ./paper.tex

Interpreted functions are often axiomatized in a definitional form, meaning that given these axioms for a well-founded function $f$, any occurrence of $f$ in a ground function term can be completely eliminated. In this paper, we consider well-founded (terminating) functions that are \textit{not mutually-recursive}. Although handling \textit{recursive} functions needs a bit of precaution because expanding any non-ground recursive function term into its definition may lead to a rewrite loop, we treat non-recursive functions the same since they can be thought of as functions that contain only base cases.
\subsection{Generating induction formulas from function definitions}\label{sec:generating}
In this section, we look at well-defined and well-founded function definitions and use them to obtain induction formulas. A function definition is a set of axioms of the form
$$F\rightarrow f(s_1,...,s_n):=t$$
each containing one branch of execution for $f$ where $F$ is the guard condition for this branch and $s_1,...,s_n$ are constructor terms or variables. For now, we treat the detection of such function definitions as a black box and address this issue later. An argument position $1\le i\le n$ of $f$ is an \textit{active argument position} if there is an axiom like above of $f$ s.t. for some $f(t_1,...,t_n)\trianglelefteq t$ and $s_i\neq t_i$. Given a literal $L$, we mark its \textit{active subterms} inductively as follows:
\begin{itemize}
	\item if $L$ is an equality $l=r$, then $l$ and $r$ are active subterms
	\item if $L$ is a predicate $p(s_1,...,s_n)$ with active argument positions $I$, then (1) if $I\neq\emptyset$ all $s_i$ with $i\in I$ are active subterms of $L$, (2) otherwise all $s_1,...,s_n$ are active subterms of $L$
	\item if $f(s_1,...,s_n)$ is an active subterm of $L$ for an $f$ with active argument positions $I$, then (1) if $I\neq\emptyset$ all $s_i$ with $i\in I$ are active subterms of $L$, (2) otherwise all $s_1,...,s_n$ are active subterms of $L$
	\item if $c(t_1,...,t_m)$ is an active constructor subterm of $L$ of type $\tau$, all $t_j$ with type $\tau$ are active subterms of $L$
\end{itemize}
\todo{should we exclude non-recursive function terms here? it is sometimes good to induct on them (which is just a case distinction) but otherwise it can be confusing to introduce this here}

Every active subterm of a literal is a potential candidate for induction since replacing these terms with some constructor term can start a cascade of simplifications, which is needed to solve base cases and to eventually use induction hypotheses in step cases. Functions can recurse on multiple arguments, so instead of create an induction formula for each active subterm, we look at each active function subterm $f(s_1,...,s_n)$ that have a non-empty active argument position set $I$ and create induction formulas from their active subterms.

In order to create an induction formula, we create (1) a case distinction on the induction terms that is exhaustive and non-overlapping (and can contain guard conditions) and (2) induction hypotheses for the step cases. For both, we create substitutions that map each induction term to some constructor term in the current case and induction hypothesis. For the case distinction, we use the function axioms which are by assumption well-defined.

Sometimes function terms contain only complex terms in their active argument positions and generalizing over them does not always lead to a proof. E.g. in $s\leq t+s$, both arguments of the outermost function $\leq$ are active but only the first argument of $+$ is active, therefore it is essential to induct on both $s$ and $t$. The other possibility is to induct on $t+s$ from the second position which would render the original true literal false.

To solve this issue, we create tuples of possible induction terms for each such function term by taking every combination of active subterms of $s_i$ for all $i\in I$ (the set of active argument positions). In the above example, this yields $(s,t)$ and $(s,t+s)$ as possible induction term tuples. We can create from them and the axioms of $\leq$ the following case distinctions by mapping each tuple element to the corresponding argument in the function's axioms:
$$\begin{aligned}
&\forall x.\textcolor{blue}{0}\leq \textcolor{red}{x}:=\top\quad&\Rightarrow\quad&\{s\mapsto \textcolor{blue}{0}, t\mapsto \textcolor{red}{x}\}\\
&\forall x.\textcolor{blue}{\suc(x)}\leq \textcolor{red}{0}:=\bot\quad&\Rightarrow\quad&\{s\mapsto \textcolor{blue}{\suc(x)}, t\mapsto \textcolor{red}{0}\},\\
&\forall x,y.\textcolor{blue}{\suc(x)}\leq \textcolor{red}{\suc(y)}:=x\leq y\quad&\Rightarrow\quad&\{s\mapsto \textcolor{blue}{\suc(x)}, t\mapsto \textcolor{red}{\suc(y)}\}
\end{aligned}$$
The next step is to add induction hypotheses where possible -- an axiom $F\rightarrow f(x_1,...,x_n):=t$ gives induction hypotheses for the corresponding case in the case distinction if there are recursive calls in $t$. Each such recursive call gives rise to a substitution the same way as in the case distinctions. For $\leq$, only the last case has a recursive call:
$$\begin{aligned}
&\forall x,y.\suc(x)\leq \suc(y):=\textcolor{blue}{x}\leq \textcolor{red}{y}\quad&\Rightarrow\quad&\{s\mapsto \textcolor{blue}{x}, t\mapsto \textcolor{red}{y}\}
\end{aligned}$$

The obtained substitutions are then applied to the original literal to create the induction formula. The case distinction is the antecedent to the final induction formula where we add a conclusion with each induction term replaced with a fresh variable:
$$\begin{pmatrix}0\leq x+0\land\forall x.(\suc(x)\leq 0+\suc(x))\land\\
\forall x,y.(x\leq y+x\rightarrow \suc(x)\leq \suc(y)+\suc(x))\end{pmatrix}\rightarrow \forall u,v.u\leq v+u$$

The substitutions obtained this way may not be confluent or well-defined. For example, substituting different terms for the same induction term only works if the substituted terms can be unified and each will be matched by the function definitions of the current function term after applying the subtitution. If this case has induction hypotheses, the unifications are applied also on the substituted terms for the same induction term in them. It may be that some cases or induction hypotheses are not unifiable this way -- any such cases or hypotheses are discarded. Losing hypotheses this way is not an issue, but losing a case from the induction formula can result in unsoundness.

\begin{example}
	A function term $f(t,t)$ and axiom $f(\bnode(\bnil,y,z),\bnode(u,v,w)):=g(f(z,u))$ gives a case where both $\bnode(\bnil,y,z)$ and $\bnode(u,v,w)$ would be substituted for $t$. This conflict can be resolved by using the unification of the two $\bnode(\bnil,y,z)$. However, in the recursive call, after applying the unification on $z$ and $u$, they are still not the same, so the induction hypothesis is discarded.
\end{example}

\subsection{Containment and union of induction formulas}
Even though there may be multiple induction formulas used for the proof of an inductive goal, the process described in Section \ref{sec:generating} can create many similar or redundant ones. Therefore an important step is to discard the unusable ones and combine the rest if possible. Looking at the Example \ref{ex:1} once again, from the literal $\neg\even(\sigma_0+\sigma_1)$ we can create two induction formulas with Skolem induction terms. One is generated from $\even(\sigma_0+\sigma_1)$:
\begin{equation}\label{eq:even1}\begin{pmatrix}\even(0+\sigma_1)\land\even(\suc(0)+\sigma_1)\land\\
\forall z.(\even(z+\sigma_1)\rightarrow\even(\suc(\suc(z))+\sigma_1))\end{pmatrix}\rightarrow \forall u.\even(u+\sigma_1)\end{equation}
The other is given by $\sigma_0+\sigma_1$:
\begin{equation}\label{eq:even2}\begin{pmatrix}\even(0+\sigma_1)\land\forall x.(\even(x+\sigma_1)\rightarrow\even(\suc(x)+\sigma_1))\end{pmatrix}\rightarrow \forall z.\even(z+\sigma_1)\end{equation}
Practically, trying the second formula is a waste of resources as we have already seen in Section \ref{sec:motivating} so we better off discarding it. There are very simple approaches in the literature on which one can base this decision. E.g. ACL2 would discard it because the step case of the first formula is in the \textit{transitive closure} of that of the second. In terms of well-founded relations, the second induction formula is based on the relation $x\prec\suc(x)$ and applying this twice gives us $x\prec\cdot\prec\suc(\suc(x))$, that is, the relation for the first formula.

The reason this works is only incidental -- although the case distinction of $+$ is more general than that of $\even$ and therefore every term in the case distinction of $\even$ will match a function definition of +, the induction hypothesis of the two are completely unrelated and therefore discarding one of them is like flipping a coin before the proof without knowing which one will be needed.

In order not to lose useful structure from the generated induction formulas, our goals are the following: when discarding or combining induction formulas (1) more special case distinctions must be preserved so that all considered function definitions are matched and (2) all induction hypotheses must be preserved since a priori we do not know which ones will be used.

For two induction formulas $F_1$ and $F_2$ with identical case distinctions, $F_1$ \textit{contains} $F_2$ if for each case, the set of induction hypotheses of the case in $F_1$ is a superset of that of $F_2$. If an induction formula is contained by another, we can safely discard it since its case distinction and induction hypotheses can be found in the second.
\begin{example}
\end{example}

If two induction formulas do not contain each other, we can either use them separately or use their \texttt{union} instead. The union of two induction formulas provides a case distinction with all cases from the original ones while supplying the necessary induction hypotheses. Given two induction formulas with cases $\mathcal{C}=\{C_i \mid 1\le i\le n\}$ and $\mathcal{D}=\{D_j \mid 1\le j\le m\}$, we can compute the union with the following two approaches:
\begin{enumerate}
	\item We can think of the two in terms of well-founded orders and for each pair of subrelations from the first and the second, we compute their intersections. Then, we check whether any subrelation has cases left "outside" the intersections. Finally, we convert this union of orders back to a formula by adding the remaining cases, i.e. the base cases.
	\item Otherwise, we can simply intersect every case of the two induction formulas putting together their induction hypotheses.
\end{enumerate}
\begin{example}
	Let us look at the case distinction of formulas \eqref{eq:even1} and \eqref{eq:even2} from our running Example \ref{ex:1}.

	\textit{Approach 1.} We convert both formulas to relations $z\prec\suc(\suc(z))$ and $x\prec\suc(x)$, respectively. The intersection of the two is a relation $z,\suc(z)\prec\suc(\suc(z))$. This can be seen as specializing the more general $\suc(x)$ to match only $\suc(\suc(z))$ and putting together the related terms $z$ and $x$, specializing them if needed -- $x$ is specialized to $\suc(z)$.

	Then, we need to check what remains of the original subrelations apart from the intersection. $z\prec\suc(\suc(z))$ is entirely covered by the intersection, this gives no further subrelations. $x\prec\suc(x)$, however, was specialized in the intersection, thus the case where $x$ is 0 is not covered -- this results in $0\prec\suc(0)$.

	Converting this back to an induction formula, to make sure the case distinction is complete, we add base case 0 which is not covered by $\suc(0)$ nor $\suc(\suc(z))$.

	\textit{Approach 2.} We take intersections of each case from \eqref{eq:even1} and \eqref{eq:even2} pairwise:
	\begin{itemize}
		\item $0$ and $0$ overlap in $0$. Both are base cases so this results in a base case.
		\item $0$ and $\suc(x)$ do not overlap.
		\item $\suc(0)$ and $0$ do not overlap.
		\item $\suc(0)$ and $\suc(x)$ overlap in $\suc(0)$. The second is a recursive case, so we add the induction hypothesis $0$ which is $x$ specialized to match the overlap.
		\item $\suc(\suc(z))$ and $0$ do not overlap.
		\item $\suc(\suc(z))$ and $\suc(x)$ overlap. Both are recursive cases, so we add the two induction hypotheses, specialized where needed ($x$ is specialized to $\suc(z)$).
	\end{itemize}

	Both approaches give the induction formula:
	\begin{equation}\begin{pmatrix}\even(0+\sigma_1)\land(\even(0+\sigma_1)\rightarrow\even(\suc(0)+\sigma_1))\land\\
	\forall z.((\even(z+\sigma_1)\land\even(\suc(z)+\sigma_1))\rightarrow\even(\suc(\suc(z))+\sigma_1))\end{pmatrix}\rightarrow \forall u.\even(u+\sigma_1)\end{equation}
\end{example}

We note here that it is enough to check pairwise intersections only because the induction formulas (and the functions they are generated from) are thought to be well-defined, so each case distinction is non-overlapping and exhaustive.

One of the advantages of taking the union is that by definition it is guaranteed that each case matches exactly one function axiom for each of the original function terms and the necessary induction hypotheses are also always present. The disadvantages are that it may not be well-founded and that it can create very large induction formulas. It is also not clear when to use it. There are situations when two induction formulas use completely different induction terms. For example, the commutativity of +
$$\forall x,y.x+y=y+x$$
can be solved by first inducting on $x$ and then on $y$ or vice versa -- the order is irrelevant. However, in general, depending on the order of simplifications and the axioms used for each function, inducting sequentially on multiple terms can lead to different results.

\subsection{Well-foundedness and well-definedness}
So far we assumed that every function we are provided with is well-founded and well-defined. Functions that fail to meet this requirement are potential sources of unsoundness. We chose easy-to-check sufficient conditions for both. We base our well-foundedness check on a lexicographic order on the active argument positions of each function and use the subterm relation as elementary property. This can be easily checked with e.g.constructor-style input functions where arguments of inductive sorts on the left-hand side of a function definition are either variables or term algebra constructors. A common example for this is the Ackermann function given by:
$$\forall y. \ack(0,y)=\suc(y)$$
$$\forall x.\ack(\suc(x),0):=\ack(x,\suc(0))$$
$$\forall x,y.\ack(\suc(x),\suc(y)):=\ack(x,\ack(\suc(x),y))$$
Any function term stemming from a function that is not well-founded is ignored while generating induction formulas.

For well-definedness, we distinguish \textit{under-definedness} and \textit{over-definedness}. The former can lead to missing false cases, thus introducing unsoundness but it can be easily mitigated by adding those missing cases. The latter is a bit more complicated: an induction formula with overlapping cases is not necessarily unsound but it can lead to ill-constructed case distinctions for the union. \todo{anything to add here?}


\subsection{Induction hypothesis strengthening}
\subsection{Generalizing over occurrences and complex terms}

\section{Function definition rewriting in Superposition}
\label{sec:fn_def_rewriting}
% !TeX root = ./paper.tex

As discussed in Section \ref{sec:motivating}, any function definition $f(\overline{x}):= t$ may be oriented in a different way in a superposition theorem prover due to $t\succ f(\overline{x})$ given by the ordering. The choices we have here is either not being able to simplify induction formulas or losing completeness.

One solution is to choose an ordering that orients all function definitions according to their intended usage \cite{terminating,kbodecidability,kbopolynomial}. However, even if the initial orientation exists, there may be new function definitions derived from the initial ones that need to be oriented on the fly which is not practically achievable. Otherwise, conflicting orientations of two variants for the same function definition can cause an infinite chain of rewriting.

Another solution is to allow rewriting according to the intended function definition orientation and at the same time disallow any inferences that would rewrite the function definitions themselves, creating new variants. Despite losing completeness in this case, depending on factors like whether we use constructor- or destructor-style function definitions, it can actually help discard inferences which we would not be used anyways, e.g. any instantiated variant for a function definition just creates redundancy in the search space.

We use this latter solution and we disallow any inferences with such function definition clauses except for rewriting with two special inference rules: one is a modified \textit{paramodulation}, the other is a special case of \textit{demodulation}. We let the special demodulation happen only if the function definition is a unit clause and it has an orientation which is simplifying w.r.t. the ordering $\succ$:
\begin{equation}
	\infer[\small{(\texttt{DemF})}]{L[t\theta]\lor D}
	{f(\overline{x}):=t & \cancel{L[f(\overline{x})\theta]\lor D}}
\end{equation}
where $f(\overline{x})\theta\succ t\theta$ and $L[f(\overline{x})\theta]\lor D\succ f(\overline{x})\theta=t\theta$. Since for functions containing no mutually recursive functions, the function "dependency tree" is a DAG, it is expected that this rule can be used at least for the base cases of any function definition since those definitions can be oriented in the right way by simply creating a precedence from the DAG.

Second, all other non-simplifying rewriting with function definitions is done with the modified paramodulation rule:
\begin{equation}
\infer[\small{(\texttt{ParF})}]{L[t\theta]\lor C\theta\lor D}{f(\overline{x}):=t\lor C & L[f(\overline{x})\theta]\lor D}
\end{equation}
This rule has no side conditions so that we can rewrite any part of a clause which helps expand function headers when needed but at the same time we lose properties such as fairness and completeness.

Note that we only use clauses for such rewriting that contain exactly one equality literal marked as a function definition. One might define e.g. non-determinism with clauses containing more than one equality:
$$\mathtt{f}(\overline{x})=t_1\lor \mathtt{f}(\overline{x})=t_2$$
We avoid handling such clauses in this manner as it is unclear how the other literal should be used in the resulting clause when we paramodulate with one of them.
%\begin{example}{continued}
%	In Example \ref{flatten-example}, the two functions \fltn and $\fltn_2$ give the four definitional axioms:
%	$$\fltn(\bnil)=\nil$$
%	$$\forall x,y,z.\fltn(\bnode(x,y,z))=\app(\fltn(x),\cons(y,\fltn(z)))$$
%	$$\forall y. \fltn_2(\bnil,y)=y$$
%	$$\forall x,y,z,u.\fltn_2(\bnode(x,y,z),u)=\fltn_2(x,\cons(y,\fltn_2(z, u)))$$
%	The base cases for the two functions can be easily oriented the "right way", i.e. to rewrite any terms such as $\fltn(\bnil)$ into \nil or $\fltn_2(\bnil,y)$ into $y$, the first by making \fltn or \bnil greater than \nil in the precedence accompanying the ordering, the second by applying the subterm property of a simplification ordering. The recursive cases, however, cannot be oriented in such a way easily, so e.g. the refutational variant of the induction step consequent from Example \ref{flatten-example}
%	$$\app(\fltn(\bnode(\sigma_0,\sigma_1,\sigma_2)),\sigma_3)\neq\fltn_2(\bnode(\sigma_0,\sigma_1,\sigma_2),\sigma_3)$$
%	would not be rewritten normally and the induction gets stuck. Instead, applying the rule (\func{ParF}) with the recursive axioms, we match the terms $\fltn(\bnode(\sigma_0,\sigma_1,\sigma_2)$ with $\fltn(\bnode(x,y,z))$ and $\fltn_2(\bnode(\sigma_0,\sigma_1,\sigma_2),\sigma_3)$ with $\fltn_2(\bnode(x,y,z),u)$ with substitution $\{x\mapsto \sigma_0,y\mapsto \sigma_1, z\mapsto\sigma_2, u\mapsto \sigma_3\}$.
%\end{example}


\section{Induction in Saturation-based proof search}

\section{Experiments}
\label{sec:experiments}
% !TeX root = ./paper.tex


\begin{figure}
	\centering
	\begin{tabular}[c]{c|c|c|c|c|c|c|c}
		\multicolumn{3}{c}{ } & \rot{30}{\textsc{Vampire}} & \rot{30}{$\textsc{Vampire}^{\ast\ast}$} & \rot{30}{$\textsc{Vampire}^{\dag}$} & \rot{30}{\textsc{CVC4}} & \rot{30}{\textsc{ZipperPosition}} \\ \hline
		\multicolumn{2}{c|}{Problems} & Count & \multicolumn{5}{c}{Solved problems} \\ \hline
		\multirow{3}{*}{$\lst$} & $\append$ & 313 & \cellcolor{blue!11!white}72 & \cellcolor{blue!38!white}238 & \cellcolor{blue!50!white}313 & \cellcolor{blue!0!white}1 & \cellcolor{blue!49!white}312\\
		& \texttt{pref} & 696 & \cellcolor{blue!6!white}94 & \cellcolor{blue!49!white}687 & \cellcolor{blue!50!white}696 & \cellcolor{blue!0!white}1 & \cellcolor{blue!41!white}583\\
		& $\rev$ & 4 & \cellcolor{blue!25!white}2 & \cellcolor{blue!25!white}2 & \cellcolor{blue!37!white}3 & \cellcolor{blue!0!white}0 & \cellcolor{blue!25!white}2\\
		\hline
		\multirow{9}{*}{$\nat$} & + assoc & 312 & \cellcolor{blue!12!white}81 & \cellcolor{blue!41!white}261 & \cellcolor{blue!50!white}312 & \cellcolor{blue!0!white}1 & \cellcolor{blue!50!white}312\\
		& + mix & 830 & \cellcolor{blue!6!white}103 & \cellcolor{blue!15!white}263 & \cellcolor{blue!35!white}591 & \cellcolor{blue!4!white}70 & \cellcolor{blue!33!white}552\\
		& + comm & 2 & \cellcolor{blue!50!white}2 & \cellcolor{blue!50!white}2 & \cellcolor{blue!50!white}2 & \cellcolor{blue!25!white}1 & \cellcolor{blue!50!white}2\\
		& $+ \suc 0$ mix & 486 & \cellcolor{blue!8!white}81 & \cellcolor{blue!12!white}124 & \cellcolor{blue!28!white}281 & \cellcolor{blue!8!white}87 & \cellcolor{blue!19!white}186\\
		& $\cdot$ & 6 & \cellcolor{blue!0!white}0 & \cellcolor{blue!0!white}0 & \cellcolor{blue!16!white}2 & \cellcolor{blue!0!white}0 & \cellcolor{blue!0!white}0\\
		& $\even$ & 2 & \cellcolor{blue!0!white}0 & \cellcolor{blue!0!white}0 & \cellcolor{blue!25!white}1 & \cellcolor{blue!0!white}0 & \cellcolor{blue!0!white}0\\
		& $\simeq$ & 4 & \cellcolor{blue!25!white}2 & \cellcolor{blue!37!white}3 & \cellcolor{blue!37!white}3 & \cellcolor{blue!25!white}2 & \cellcolor{blue!50!white}4\\
		& $\leq$ & 696 & \cellcolor{blue!8!white}117 & \cellcolor{blue!50!white}696 & \cellcolor{blue!50!white}696 & \cellcolor{blue!0!white}1 & \cellcolor{blue!41!white}583\\
		& $\dup$ & 1 & \cellcolor{blue!0!white}0 & \cellcolor{blue!0!white}0 & \cellcolor{blue!0!white}0 & \cellcolor{blue!0!white}0 & \cellcolor{blue!0!white}0\\
		\hline
		\multirow{1}{*}{$\btree$} & $\fltn$ & 3 & \cellcolor{blue!0!white}0 & \cellcolor{blue!0!white}0 & \cellcolor{blue!33!white}2 & \cellcolor{blue!16!white}1 & \cellcolor{blue!33!white}2\\
		\hline
		\multirow{1}{*}{$\nat\&\lst$} & $+\&\append$ & 1 & \cellcolor{blue!50!white}1 & \cellcolor{blue!50!white}1 & \cellcolor{blue!50!white}1 & \cellcolor{blue!0!white}0 & \cellcolor{blue!0!white}0\\
		\hline
		\multicolumn{2}{c|}{UFDTLIA} & 327 & \cellcolor{blue!22!white}144 & \cellcolor{blue!22!white}149 & \cellcolor{blue!26!white}176 & \cellcolor{blue!32!white}214 & -\\
		\hhline{|=|=|=|=|=|=|=|=|}
		\multicolumn{2}{c|}{All} & 3683 & \cellcolor{blue!9!white}699 & \cellcolor{blue!33!white}2426 & \cellcolor{blue!42!white}3079 & \cellcolor{blue!5!white}379 & \cellcolor{blue!38!white}2538\\
		\hline
		\multicolumn{8}{c}{ }\\
		\hline
		\multicolumn{2}{c|}{Problems} & Count & \multicolumn{5}{c}{Uniquely solved problems} \\ \hline
		\multirow{3}{*}{$\lst$} & $\append$ & 313 & 0 & 0 & \cellcolor{gray!20!white}1 & 0 & 0\\
		& $\pref$ & 696 & 0 & 0 & 0 & 0 & 0\\
		& $\rev$ & 4 & 0 & 0 & \cellcolor{gray!20!white}1 & 0 & 0\\
		\hline
		\multirow{9}{*}{$\nat$} & + assoc & 312 & 0 & 0 & 0 & 0 & 0\\
		& + mix & 830 & \cellcolor{gray!20!white}2 & 0 & \cellcolor{gray!20!white}116 & 0 & \cellcolor{gray!20!white}93\\
		& + comm & 2 & 0 & 0 & 0 & 0 & 0\\
		& $+ \suc 0$ mix & 486 & 0 & 0 & \cellcolor{gray!20!white}95 & 0 & \cellcolor{gray!20!white}3\\
		& $\cdot$ & 6 & 0 & 0 & \cellcolor{gray!20!white}2 & 0 & 0\\
		& $\even$ & 2 & 0 & 0 & \cellcolor{gray!20!white}1 & 0 & 0\\
		& $\simeq$ & 4 & 0 & 0 & 0 & 0 & 0\\
		& $\leq$ & 696 & 0 & 0 & 0 & 0 & 0\\
		& $\dup$ & 1 & 0 & 0 & 0 & 0 & 0\\
		\hline
		\multirow{1}{*}{$\btree$} & $\fltn$ & 3 & 0 & 0 & 0 & 0 & 0\\
		\hline
		\multirow{1}{*}{$\nat\&\lst$} & $+\&\append$ & 1 & 0 & 0 & 0 & 0 & 0\\
		\hline
		\multicolumn{2}{c|}{UFDTLIA} & 327 & \cellcolor{gray!20!white}2 & 0 & \cellcolor{gray!20!white}9 & \cellcolor{gray!20!white}50 & -\\
		\hhline{|=|=|=|=|=|=|=|=|}
		\multicolumn{2}{c|}{All} & 3683 & \cellcolor{gray!20!white}4 & 0 & \cellcolor{gray!20!white}225 & \cellcolor{gray!20!white}50 & \cellcolor{gray!20!white}96\\
		\hline
	\end{tabular}
	\caption{\textsc{Vampire} with and without portfolio mode and other solvers compared}
	\label{fig:comparison3}
\end{figure}


\section{Related Work}
\input{related_work.tex}

\section{Conclusions}

\section*{Acknowledgements}

%
% ---- Bibliography ----
%
% BibTeX users should specify bibliography style 'splncs04'.
% References will then be sorted and formatted in the correct style.
%
% \bibliographystyle{splncs04}
% \bibliography{mybibliography}
%

\bibliographystyle{splncs04}
\bibliography{bib}
 
\end{document}

