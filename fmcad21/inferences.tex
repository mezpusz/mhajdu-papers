% !TeX root = ./paper.tex

\subsection{Proving equational goals}
In general, proving first-order theorems regarding inductive types requires not only using inference rules, but also coming up with the "right" induction formula when needed which is known to be undecidable.

Moreover, as suggested by Example \ref{ex:1}, induction and the superposition calculus seem to be incompatible since we have to sometimes go against the ordering to get a proof. This apparent incompatibility does not contradict any completeness results and comes rather from the automation of induction -- if we know \textit{a priori} what induction formulas to use, putting them into the search space and letting any standard superposition prover do its work will give a proof.

In this part, we focus on describing how function definitions and induction hypotheses of equational induction formulas are treated in a special way to get more proofs.

\subsubsection{Function definitions}
We assume that any function definition axiom is of the form:
$$C\lor \mathtt{f}(\overline{x}) \rightsquigarrow t$$

On a high-level, we follow the induction inferences of \cite{vampireinduction,vampiregeneralization}. We will extend this 