

\documentclass[conference]{IEEEtran}

\usepackage{vampire_induction}
\usepackage{amsmath}
\usepackage{mathtools}
\usepackage{amssymb}

\newtheorem{example}{Example}

% correct bad hyphenation here
\hyphenation{op-tical net-works semi-conduc-tor}

\begin{document}
	\title{Recursive Functions in Saturation}

%	\author{\IEEEauthorblockN{Márton Hajdu}
%		\IEEEauthorblockA{TU Wien}
%		\and
%		\IEEEauthorblockN{Petra Hozzová}
%		\IEEEauthorblockA{TU Wien}
%		\and
%		\IEEEauthorblockN{Laura Kovács}
%		\IEEEauthorblockA{TU Wien}
%		\and
%		\IEEEauthorblockN{Andrei Voronkov}
%		\IEEEauthorblockA{University of Manchester}}

	\author{\IEEEauthorblockN{Márton Hajdu\IEEEauthorrefmark{1},
	Petra Hozzová\IEEEauthorrefmark{1},
	Laura Kovács\IEEEauthorrefmark{1} and
	Andrei Voronkov\IEEEauthorrefmark{2}}
	\IEEEauthorblockA{\IEEEauthorrefmark{1}TU Wien}
	\IEEEauthorblockA{\IEEEauthorrefmark{2}University of Manchester}}

	\maketitle

	% As a general rule, do not put math, special symbols or citations
	% in the abstract
	\begin{abstract}
		The abstract goes here.
	\end{abstract}
	
	% no keywords
	
	
	
	
	% For peer review papers, you can put extra information on the cover
	% page as needed:
	% \ifCLASSOPTIONpeerreview
	% \begin{center} \bfseries EDICS Category: 3-BBND \end{center}
	% \fi
	%
	% For peerreview papers, this IEEEtran command inserts a page break and
	% creates the second title. It will be ignored for other modes.
	\IEEEpeerreviewmaketitle
	
	
	
	\section{Introduction}
	% no \IEEEPARstart
	This demo file is intended to serve as a ``starter file''
	for IEEE conference papers produced under \LaTeX\ using
	IEEEtran.cls version 1.8b and later.
	% You must have at least 2 lines in the paragraph with the drop letter
	% (should never be an issue)
	I wish you the best of success.
	
	\hfill mds
	
	\hfill August 26, 2015
	
	\subsection{Subsection Heading Here}
	Subsection text here.
	
	
	\subsubsection{Subsubsection Heading Here}
	Subsubsection text here.
	
	
	% An example of a floating figure using the graphicx package.
	% Note that \label must occur AFTER (or within) \caption.
	% For figures, \caption should occur after the \includegraphics.
	% Note that IEEEtran v1.7 and later has special internal code that
	% is designed to preserve the operation of \label within \caption
	% even when the captionsoff option is in effect. However, because
	% of issues like this, it may be the safest practice to put all your
	% \label just after \caption rather than within \caption{}.
	%
	% Reminder: the "draftcls" or "draftclsnofoot", not "draft", class
	% option should be used if it is desired that the figures are to be
	% displayed while in draft mode.
	%
	%\begin{figure}[!t]
	%\centering
	%\includegraphics[width=2.5in]{myfigure}
	% where an .eps filename suffix will be assumed under latex, 
	% and a .pdf suffix will be assumed for pdflatex; or what has been declared
	% via \DeclareGraphicsExtensions.
	%\caption{Simulation results for the network.}
	%\label{fig_sim}
	%\end{figure}
	
	% Note that the IEEE typically puts floats only at the top, even when this
	% results in a large percentage of a column being occupied by floats.
	
	
	% An example of a double column floating figure using two subfigures.
	% (The subfig.sty package must be loaded for this to work.)
	% The subfigure \label commands are set within each subfloat command,
	% and the \label for the overall figure must come after \caption.
	% \hfil is used as a separator to get equal spacing.
	% Watch out that the combined width of all the subfigures on a 
	% line do not exceed the text width or a line break will occur.
	%
	%\begin{figure*}[!t]
	%\centering
	%\subfloat[Case I]{\includegraphics[width=2.5in]{box}%
	%\label{fig_first_case}}
	%\hfil
	%\subfloat[Case II]{\includegraphics[width=2.5in]{box}%
	%\label{fig_second_case}}
	%\caption{Simulation results for the network.}
	%\label{fig_sim}
	%\end{figure*}
	%
	% Note that often IEEE papers with subfigures do not employ subfigure
	% captions (using the optional argument to \subfloat[]), but instead will
	% reference/describe all of them (a), (b), etc., within the main caption.
	% Be aware that for subfig.sty to generate the (a), (b), etc., subfigure
	% labels, the optional argument to \subfloat must be present. If a
	% subcaption is not desired, just leave its contents blank,
	% e.g., \subfloat[].
	
	
	% An example of a floating table. Note that, for IEEE style tables, the
	% \caption command should come BEFORE the table and, given that table
	% captions serve much like titles, are usually capitalized except for words
	% such as a, an, and, as, at, but, by, for, in, nor, of, on, or, the, to
	% and up, which are usually not capitalized unless they are the first or
	% last word of the caption. Table text will default to \footnotesize as
	% the IEEE normally uses this smaller font for tables.
	% The \label must come after \caption as always.
	%
	%\begin{table}[!t]
	%% increase table row spacing, adjust to taste
	%\renewcommand{\arraystretch}{1.3}
	% if using array.sty, it might be a good idea to tweak the value of
	% \extrarowheight as needed to properly center the text within the cells
	%\caption{An Example of a Table}
	%\label{table_example}
	%\centering
	%% Some packages, such as MDW tools, offer better commands for making tables
	%% than the plain LaTeX2e tabular which is used here.
	%\begin{tabular}{|c||c|}
	%\hline
	%One & Two\\
	%\hline
	%Three & Four\\
	%\hline
	%\end{tabular}
	%\end{table}
	
	
	% Note that the IEEE does not put floats in the very first column
	% - or typically anywhere on the first page for that matter. Also,
	% in-text middle ("here") positioning is typically not used, but it
	% is allowed and encouraged for Computer Society conferences (but
	% not Computer Society journals). Most IEEE journals/conferences use
	% top floats exclusively. 
	% Note that, LaTeX2e, unlike IEEE journals/conferences, places
	% footnotes above bottom floats. This can be corrected via the
	% \fnbelowfloat command of the stfloats package.
	
	\section{Motivating Examples}
	\label{sec:motivating}
	% !TeX root = ./paper.tex

\begin{figure}
\footnotesize
%\begin{minipage}[t]{0.4\textwidth}
%$$\begin{aligned}&\forall y.0+y:=y\\
%&\forall x,y.\suc(x)+y:=\suc(x+y)\\
%\end{aligned}$$
%\hrule
%$$\begin{aligned}&\even(0):=\top\\
%&\even.2:\even(\suc(0)):=\bot\\
%&\forall z.\even(\suc(\suc(z))):=\even(z)\\
%\end{aligned}$$
%\hrule
%$$\begin{aligned}&\forall x.0\leq x:=\top\\
%&\forall x.\suc(x)\leq 0:=\bot\\
%&\forall x,y.\suc(x)\leq \suc(y):=x\leq y\\
%\end{aligned}$$\end{minipage}
%\begin{minipage}[t]{0.5\textwidth}
%$$\begin{aligned}&\forall x.\nil\append x:=x\\
%&\forall x,y,z.\cons(x,y)\append z:=\cons(x,y\append z)\\
%\end{aligned}$$
%\hrule
%$$\begin{aligned}
%&\fltn(\bnil):=\nil\\
%&\forall u,v,w.\fltn(\bnode(u,v,w)):=\\&\quad\fltn(u)\append\cons(v,\fltn(w))
%\end{aligned}$$
%\hrule
%$$\begin{aligned}
%&\forall x.\fltn_2(\bnil,x):=x\\
%&\forall u,v,w,y.\fltn_2(\bnode(u,v,w),y):=\\&\quad\fltn_2(u,\cons(v,\fltn_2(w,y)))$$
%\end{minipage}
$$\begin{aligned}&\forall y.0+y&&:=y&&\forall x.\nil\append x&&:=x\\
&\forall x,y.\suc(x)+y&&:=\suc(x+y)&&\forall x,y,z.\cons(x,y)\append z&&:=\cons(x,y\append z)\\
\cline{1-8}
&\even(0)&&:=\top&&\fltn(\bnil)&&:=\nil\\
&\even(\suc(0))&&:=\bot&&\forall u,v,w.\fltn(\bnode(u,v,w))&&:=\\
&\forall z.\even(\suc(\suc(z)))&&:=\even(z)&&\phantom{aa}\fltn(u)\append\cons(v,\fltn(w))\\
\cline{1-8}
&\forall x.0\leq x&&:=\top&&\forall x.\fltn_2(\bnil,x)&&:=x\\
&\forall x.\suc(x)\leq 0&&:=\bot&&\forall u,v,w,y.\fltn_2(\bnode(u,v,w),y)&&:=\\
&\forall x,y.\suc(x)\leq \suc(y)&&:=x\leq y&&\phantom{aa}\fltn_2(u,\cons(v,\fltn_2(w,y)))\\
\end{aligned}$$
\caption{Recursive function definitions used}
\label{fig:functions}
\end{figure}
\normalsize

Finding an induction formula that leads to a successful proof for an inductive problem is known to be undecidable. Some provers like \textsc{ACL2} or \textsc{IsaPlanner} take into account the structure of any function present in an inductive goal when creating induction formulas and provide heuristics to select the most suitable one as the next proof step. Such heuristics are however still relatively uncommon in saturation-based theorem provers.
\begin{example}\label{ex:1}
$$\forall x,y. (\even(x)\land \even(y))\rightarrow \even(x+y)$$
The function definitions for $\even$ and + can be found in Figure \ref{fig:functions}. We can do the straightforward derivation:
\footnotesize
\begin{mdframed}[usetwoside=false,innertopmargin=-5pt,skipabove=0pt,skipbelow=0pt]\begin{equation}\nonumber\begin{aligned}
&\even(\sigma_0)&&\text{1. input}\\
&\even(\sigma_1)&&\text{2. input}\\
&\neg\even(\sigma_0+\sigma_1)&&\text{3. input}\\
\hline
&\begin{pmatrix}\even(0+\sigma_1)\land\\\forall z.(\even(z+\sigma_1)\rightarrow\even(\suc(z)+\sigma_1))\end{pmatrix}\rightarrow \forall x.\even(x+\sigma_1)&&\text{4. induction f.}\\
&\neg\even(0+\sigma_1)\lor \even(\sigma_2+\sigma_1)&&\text{5. bin.res. 3, cnf(4)}\\
&\neg\even(0+\sigma_1)\lor \neg\even(\suc(\sigma_2)+\sigma_1)&&\text{6. bin.res. 3, cnf(4)}\\
&\neg\even(\sigma_1)\lor \even(\sigma_2+\sigma_1)&&\text{7. 5, + axiom}\\
&\neg\even(\sigma_1)\lor \neg\even(\suc(\sigma_2)+\sigma_1)&&\text{8. 6, + axiom}\\
&\even(\sigma_2+\sigma_1)&&\text{9. bin.res. 2, 7}\\
&\neg\even(\suc(\sigma_2)+\sigma_1)&&\text{10. bin.res. 2, 8}\\
&\neg\even(\suc(\sigma_2+\sigma_1))&&\text{11. 10, + axiom}\\
\hline
&...\\
&\even(\suc(\sigma_3+\sigma_1))&&\text{12.}\\
&\neg\even(\sigma_3+\sigma_1)&&\text{13.}
\end{aligned}\end{equation}\end{mdframed}
\normalsize
After selecting (3) for induction and binary resolving it with the clausal form of the simplest induction formula (4) with $\sigma_0$ as induction term, we get (5) and (6). Simplifications give (9) and (11) but there is no $\even$ axiom to further simplify. One more induction on $\sigma_2$ and similar simplifications yield (12) and (13). Neither of the hypotheses (9) and (12) match the conclusions (11) and (13), either due to the different Skolem constants or the different term structure.

Using $\sigma_1$ as induction term does not help either -- we cannot get rid of any constructor terms in the second argument position of +.
\end{example}
In this paper, we create "matching" induction formulas using function definitions and then -- if different induction formulas are generated for a goal -- combine these in a way that all function terms can be simplified in the case distinction while keeping the necessary hypotheses.

In a saturation-based theorem prover, even the "correct" induction formula is useless if the term ordering prohibits rewriting function terms into their definitions or using necessary induction hypothesis, ultimately leading to a stuck proof.
\begin{example}
	Given a unit-clause:
	$$\{\fltn(\sigma_0)\append\sigma_1\neq\fltn_2(\sigma_0,\sigma_1)\}$$
	Proving this literal inductively with the simplest case distinction gives a step case $\bnode(u,v,w)$ with induction hypotheses $u$ and $v$. Due to the large terms on the right-hand side of function definitions $\fltn$ and $\fltn_2$ for case $\bnode(u,v,w)$, the induction step conclusion cannot be simplified:
	$$\fltn(\bnode(\sigma_2,\sigma_3,\sigma_4))\append\sigma_1\neq\fltn_2(\bnode(\sigma_2,\sigma_3,\sigma_4),\sigma_1)$$
\end{example}
So even though the right induction formula was found we could not solve one of its cases because of limitations in the calculus. We address this issue in a way that still preserves relative completeness and some of the properties of simplifications.
	
	\section{Inferences}
	\label{sec:inferences}
	% !TeX root = ./paper.tex

\subsection{Proving equational goals}
In general, proving first-order theorems regarding inductive types requires not only using inference rules, but also coming up with the "right" induction formula when needed which is known to be undecidable.

Moreover, as suggested by Example \ref{ex:1}, induction and the superposition calculus seem to be incompatible since we have to sometimes go against the ordering to get a proof. This apparent incompatibility does not contradict any completeness results and comes rather from the automation of induction -- if we know \textit{a priori} what induction formulas to use, putting them into the search space and letting any standard superposition prover do its work will give a proof.

In this part, we focus on describing how function definitions and induction hypotheses of equational induction formulas are treated in a special way to get more proofs.

\subsubsection{Function definitions}
We assume that any function definition axiom is of the form:
$$C\lor \mathtt{f}(\overline{x}) \rightsquigarrow t$$

On a high-level, we follow the induction inferences of \cite{vampireinduction,vampiregeneralization}. We will extend this 
	
	
	
	
	\section{Conclusion}
	The conclusion goes here.
	
	
	
	
	% conference papers do not normally have an appendix
	
	
	% use section* for acknowledgment
	\section*{Acknowledgment}
	
	
	The authors would like to thank...
	
	
	
	
	
	% trigger a \newpage just before the given reference
	% number - used to balance the columns on the last page
	% adjust value as needed - may need to be readjusted if
	% the document is modified later
	%\IEEEtriggeratref{8}
	% The "triggered" command can be changed if desired:
	%\IEEEtriggercmd{\enlargethispage{-5in}}
	
	% references section
	
	% can use a bibliography generated by BibTeX as a .bbl file
	% BibTeX documentation can be easily obtained at:
	% http://mirror.ctan.org/biblio/bibtex/contrib/doc/
	% The IEEEtran BibTeX style support page is at:
	% http://www.michaelshell.org/tex/ieeetran/bibtex/
	%\bibliographystyle{IEEEtran}
	% argument is your BibTeX string definitions and bibliography database(s)
	%\bibliography{IEEEabrv,../bib/paper}
	%
	% <OR> manually copy in the resultant .bbl file
	% set second argument of \begin to the number of references
	% (used to reserve space for the reference number labels box)
	

	\bibliographystyle{IEEEtran}

	\bibliography{paper-bib}

	
%	\begin{thebibliography}{1}
%		
%		\bibitem{IEEEhowto:kopka}
%		H.~Kopka and P.~W. Daly, \emph{A Guide to \LaTeX}, 3rd~ed.\hskip 1em plus
%		0.5em minus 0.4em\relax Harlow, England: Addison-Wesley, 1999.
%		
%	\end{thebibliography}
	
	
	
	
	% that's all folks
\end{document}


