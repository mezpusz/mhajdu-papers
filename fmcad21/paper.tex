

\documentclass[conference]{IEEEtran}

\usepackage{vampire_induction}
\usepackage{amsmath}
\usepackage{mathtools}
\usepackage{amssymb}

\usepackage{proof}

\usepackage{cancel}
\usepackage{todonotes}

\newtheorem{example}{Example}

% correct bad hyphenation here
\hyphenation{op-tical net-works semi-conduc-tor}

\begin{document}
	\title{Inductive Reasoning with Recursive Function Definitions}

%	\author{\IEEEauthorblockN{Márton Hajdu}
%		\IEEEauthorblockA{TU Wien}
%		\and
%		\IEEEauthorblockN{Petra Hozzová}
%		\IEEEauthorblockA{TU Wien}
%		\and
%		\IEEEauthorblockN{Laura Kovács}
%		\IEEEauthorblockA{TU Wien}
%		\and
%		\IEEEauthorblockN{Andrei Voronkov}
%		\IEEEauthorblockA{University of Manchester}}

	\author{\IEEEauthorblockN{Márton Hajdu\IEEEauthorrefmark{1},
	Petra Hozzová\IEEEauthorrefmark{1},
	Laura Kovács\IEEEauthorrefmark{1} and
	Andrei Voronkov\IEEEauthorrefmark{2}}
	\IEEEauthorblockA{\IEEEauthorrefmark{1}TU Wien}
	\IEEEauthorblockA{\IEEEauthorrefmark{2}University of
          Manchester and EasyChair}}

	\maketitle

	% As a general rule, do not put math, special symbols or citations
	% in the abstract
	\begin{abstract}
Functional programs over inductively defined data types, such as lists and naturals,
can naturally be defined using recursive equations over recursive functions. 
In first-order logic,  function
definitions can be considered as universally quantified equalities. 
Verifying functional program properties therefore 
 requires inductive 
reasoning with both theories and quantifiers. 
In this paper we propose new  extensions and
generalizations to automate induction with recursive functions in saturation-based first-order theorem proving.
Instead of using function definitions as first-order axioms, we introduced new simplification rules for
treating function definitions as rewrite rules and guide inductive reasoning over recursive function terms.
	\end{abstract}
	
	% no keywords
	
	
	
	
	% For peer review papers, you can put extra information on the cover
	% page as needed:
	% \ifCLASSOPTIONpeerreview
	% \begin{center} \bfseries EDICS Category: 3-BBND \end{center}
	% \fi
	%
	% For peerreview papers, this IEEEtran command inserts a page break and
	% creates the second title. It will be ignored for other modes.
	\IEEEpeerreviewmaketitle
	
	% An example of a floating figure using the graphicx package.
	% Note that \label must occur AFTER (or within) \caption.
	% For figures, \caption should occur after the \includegraphics.
	% Note that IEEEtran v1.7 and later has special internal code that
	% is designed to preserve the operation of \label within \caption
	% even when the captionsoff option is in effect. However, because
	% of issues like this, it may be the safest practice to put all your
	% \label just after \caption rather than within \caption{}.
	%
	% Reminder: the "draftcls" or "draftclsnofoot", not "draft", class
	% option should be used if it is desired that the figures are to be
	% displayed while in draft mode.
	%
	%\begin{figure}[!t]
	%\centering
	%\includegraphics[width=2.5in]{myfigure}
	% where an .eps filename suffix will be assumed under latex, 
	% and a .pdf suffix will be assumed for pdflatex; or what has been declared
	% via \DeclareGraphicsExtensions.
	%\caption{Simulation results for the network.}
	%\label{fig_sim}
	%\end{figure}
	
	% Note that the IEEE typically puts floats only at the top, even when this
	% results in a large percentage of a column being occupied by floats.
	
	
	% An example of a double column floating figure using two subfigures.
	% (The subfig.sty package must be loaded for this to work.)
	% The subfigure \label commands are set within each subfloat command,
	% and the \label for the overall figure must come after \caption.
	% \hfil is used as a separator to get equal spacing.
	% Watch out that the combined width of all the subfigures on a 
	% line do not exceed the text width or a line break will occur.
	%
	%\begin{figure*}[!t]
	%\centering
	%\subfloat[Case I]{\includegraphics[width=2.5in]{box}%
	%\label{fig_first_case}}
	%\hfil
	%\subfloat[Case II]{\includegraphics[width=2.5in]{box}%
	%\label{fig_second_case}}
	%\caption{Simulation results for the network.}
	%\label{fig_sim}
	%\end{figure*}
	%
	% Note that often IEEE papers with subfigures do not employ subfigure
	% captions (using the optional argument to \subfloat[]), but instead will
	% reference/describe all of them (a), (b), etc., within the main caption.
	% Be aware that for subfig.sty to generate the (a), (b), etc., subfigure
	% labels, the optional argument to \subfloat must be present. If a
	% subcaption is not desired, just leave its contents blank,
	% e.g., \subfloat[].
	
	
	% An example of a floating table. Note that, for IEEE style tables, the
	% \caption command should come BEFORE the table and, given that table
	% captions serve much like titles, are usually capitalized except for words
	% such as a, an, and, as, at, but, by, for, in, nor, of, on, or, the, to
	% and up, which are usually not capitalized unless they are the first or
	% last word of the caption. Table text will default to \footnotesize as
	% the IEEE normally uses this smaller font for tables.
	% The \label must come after \caption as always.
	%
	%\begin{table}[!t]
	%% increase table row spacing, adjust to taste
	%\renewcommand{\arraystretch}{1.3}
	% if using array.sty, it might be a good idea to tweak the value of
	% \extrarowheight as needed to properly center the text within the cells
	%\caption{An Example of a Table}
	%\label{table_example}
	%\centering
	%% Some packages, such as MDW tools, offer better commands for making tables
	%% than the plain LaTeX2e tabular which is used here.
	%\begin{tabular}{|c||c|}
	%\hline
	%One & Two\\
	%\hline
	%Three & Four\\
	%\hline
	%\end{tabular}
	%\end{table}
	
	
	% Note that the IEEE does not put floats in the very first column
	% - or typically anywhere on the first page for that matter. Also,
	% in-text middle ("here") positioning is typically not used, but it
	% is allowed and encouraged for Computer Society conferences (but
	% not Computer Society journals). Most IEEE journals/conferences use
	% top floats exclusively. 
	% Note that, LaTeX2e, unlike IEEE journals/conferences, places
	% footnotes above bottom floats. This can be corrected via the
	% \fnbelowfloat command of the stfloats package.
	
	\section{Motivating Examples}
	\label{sec:motivating}
	% !TeX root = ./paper.tex

\begin{figure}
\footnotesize
%\begin{minipage}[t]{0.4\textwidth}
%$$\begin{aligned}&\forall y.0+y:=y\\
%&\forall x,y.\suc(x)+y:=\suc(x+y)\\
%\end{aligned}$$
%\hrule
%$$\begin{aligned}&\even(0):=\top\\
%&\even.2:\even(\suc(0)):=\bot\\
%&\forall z.\even(\suc(\suc(z))):=\even(z)\\
%\end{aligned}$$
%\hrule
%$$\begin{aligned}&\forall x.0\leq x:=\top\\
%&\forall x.\suc(x)\leq 0:=\bot\\
%&\forall x,y.\suc(x)\leq \suc(y):=x\leq y\\
%\end{aligned}$$\end{minipage}
%\begin{minipage}[t]{0.5\textwidth}
%$$\begin{aligned}&\forall x.\nil\append x:=x\\
%&\forall x,y,z.\cons(x,y)\append z:=\cons(x,y\append z)\\
%\end{aligned}$$
%\hrule
%$$\begin{aligned}
%&\fltn(\bnil):=\nil\\
%&\forall u,v,w.\fltn(\bnode(u,v,w)):=\\&\quad\fltn(u)\append\cons(v,\fltn(w))
%\end{aligned}$$
%\hrule
%$$\begin{aligned}
%&\forall x.\fltn_2(\bnil,x):=x\\
%&\forall u,v,w,y.\fltn_2(\bnode(u,v,w),y):=\\&\quad\fltn_2(u,\cons(v,\fltn_2(w,y)))$$
%\end{minipage}
$$\begin{aligned}&\forall y.0+y&&:=y&&\forall x.\nil\append x&&:=x\\
&\forall x,y.\suc(x)+y&&:=\suc(x+y)&&\forall x,y,z.\cons(x,y)\append z&&:=\cons(x,y\append z)\\
\cline{1-8}
&\even(0)&&:=\top&&\fltn(\bnil)&&:=\nil\\
&\even(\suc(0))&&:=\bot&&\forall u,v,w.\fltn(\bnode(u,v,w))&&:=\\
&\forall z.\even(\suc(\suc(z)))&&:=\even(z)&&\phantom{aa}\fltn(u)\append\cons(v,\fltn(w))\\
\cline{1-8}
&\forall x.0\leq x&&:=\top&&\forall x.\fltn_2(\bnil,x)&&:=x\\
&\forall x.\suc(x)\leq 0&&:=\bot&&\forall u,v,w,y.\fltn_2(\bnode(u,v,w),y)&&:=\\
&\forall x,y.\suc(x)\leq \suc(y)&&:=x\leq y&&\phantom{aa}\fltn_2(u,\cons(v,\fltn_2(w,y)))\\
\end{aligned}$$
\caption{Recursive function definitions used}
\label{fig:functions}
\end{figure}
\normalsize

Finding an induction formula that leads to a successful proof for an inductive problem is known to be undecidable. Some provers like \textsc{ACL2} or \textsc{IsaPlanner} take into account the structure of any function present in an inductive goal when creating induction formulas and provide heuristics to select the most suitable one as the next proof step. Such heuristics are however still relatively uncommon in saturation-based theorem provers.
\begin{example}\label{ex:1}
$$\forall x,y. (\even(x)\land \even(y))\rightarrow \even(x+y)$$
The function definitions for $\even$ and + can be found in Figure \ref{fig:functions}. We can do the straightforward derivation:
\footnotesize
\begin{mdframed}[usetwoside=false,innertopmargin=-5pt,skipabove=0pt,skipbelow=0pt]\begin{equation}\nonumber\begin{aligned}
&\even(\sigma_0)&&\text{1. input}\\
&\even(\sigma_1)&&\text{2. input}\\
&\neg\even(\sigma_0+\sigma_1)&&\text{3. input}\\
\hline
&\begin{pmatrix}\even(0+\sigma_1)\land\\\forall z.(\even(z+\sigma_1)\rightarrow\even(\suc(z)+\sigma_1))\end{pmatrix}\rightarrow \forall x.\even(x+\sigma_1)&&\text{4. induction f.}\\
&\neg\even(0+\sigma_1)\lor \even(\sigma_2+\sigma_1)&&\text{5. bin.res. 3, cnf(4)}\\
&\neg\even(0+\sigma_1)\lor \neg\even(\suc(\sigma_2)+\sigma_1)&&\text{6. bin.res. 3, cnf(4)}\\
&\neg\even(\sigma_1)\lor \even(\sigma_2+\sigma_1)&&\text{7. 5, + axiom}\\
&\neg\even(\sigma_1)\lor \neg\even(\suc(\sigma_2)+\sigma_1)&&\text{8. 6, + axiom}\\
&\even(\sigma_2+\sigma_1)&&\text{9. bin.res. 2, 7}\\
&\neg\even(\suc(\sigma_2)+\sigma_1)&&\text{10. bin.res. 2, 8}\\
&\neg\even(\suc(\sigma_2+\sigma_1))&&\text{11. 10, + axiom}\\
\hline
&...\\
&\even(\suc(\sigma_3+\sigma_1))&&\text{12.}\\
&\neg\even(\sigma_3+\sigma_1)&&\text{13.}
\end{aligned}\end{equation}\end{mdframed}
\normalsize
After selecting (3) for induction and binary resolving it with the clausal form of the simplest induction formula (4) with $\sigma_0$ as induction term, we get (5) and (6). Simplifications give (9) and (11) but there is no $\even$ axiom to further simplify. One more induction on $\sigma_2$ and similar simplifications yield (12) and (13). Neither of the hypotheses (9) and (12) match the conclusions (11) and (13), either due to the different Skolem constants or the different term structure.

Using $\sigma_1$ as induction term does not help either -- we cannot get rid of any constructor terms in the second argument position of +.
\end{example}
In this paper, we create "matching" induction formulas using function definitions and then -- if different induction formulas are generated for a goal -- combine these in a way that all function terms can be simplified in the case distinction while keeping the necessary hypotheses.

In a saturation-based theorem prover, even the "correct" induction formula is useless if the term ordering prohibits rewriting function terms into their definitions or using necessary induction hypothesis, ultimately leading to a stuck proof.
\begin{example}
	Given a unit-clause:
	$$\{\fltn(\sigma_0)\append\sigma_1\neq\fltn_2(\sigma_0,\sigma_1)\}$$
	Proving this literal inductively with the simplest case distinction gives a step case $\bnode(u,v,w)$ with induction hypotheses $u$ and $v$. Due to the large terms on the right-hand side of function definitions $\fltn$ and $\fltn_2$ for case $\bnode(u,v,w)$, the induction step conclusion cannot be simplified:
	$$\fltn(\bnode(\sigma_2,\sigma_3,\sigma_4))\append\sigma_1\neq\fltn_2(\bnode(\sigma_2,\sigma_3,\sigma_4),\sigma_1)$$
\end{example}
So even though the right induction formula was found we could not solve one of its cases because of limitations in the calculus. We address this issue in a way that still preserves relative completeness and some of the properties of simplifications.

	\section{Preliminaries}
	\label{sec:preliminaries}
	% !TeX root = ./paper.tex

We assume familiarity with \textit{standard multi-sorted first-order logic with equality}. Functions are denoted with $f$, $g$, $h$, predicates with $p$, $q$, $r$ and variables with $x$, $y$, $z$, $u$, $v$, $w$, possibly with indices. We reserve the notation $\sigma$, $\sigma_0$, $\sigma_1$, etc. for Skolem constants. A term is \textit{ground} if it contains no variables. The notation $\overline{x}$ and $\overline{t}$ means tuples of variables and terms, respectively.

We use the standard logical connectives $\neg$, $\lor$, $\land$, $\rightarrow$ and $\leftrightarrow$ and quantifiers $\forall$ and $\exists$. \textit{Atoms} are built inductively from terms and predicate symbols. Atoms and their negations are called \textit{literals}. For a literal $l$, we use the notation $\overline{l}$ to denote its opposite sign literal. \textit{Formulas} are built from connectives and atoms.

Additionally, a disjunction of literals is a \textit{clause}. We reserve the symbol $\square$ for the \textit{empty clause} which is logically equivalent to $\bot$. We call every term, literal, clause or formula an \textit{expression}. We use the notation $s\trianglelefteq t$ to denote that $s$ is a \textit{subterm} of $t$ and $s\triangleleft t$ if $s$ is a \textit{proper subterm} of $t$.

We may use the words \textit{sort} and \textit{type} interchangeably. We distinguish special sorts called \textit{inductive sorts}, function symbols for inductive sorts called \textit{constructors} and \textit{destructors}. We require that the signature contains at least one constant constructor symbol for every inductive type. Such a symbol is called a \textit{base constructor}, while non-constant ones are called \textit{recursive constructors}. We call the ground terms built from the constructor symbols of a sort its \textit{term algebra}. Semantically, each $n$-ary constructor $c$ has $n$ corresponding destructors $d_1,...,d_n$. For any constructor term $c(t_1,...,t_n)$ with root symbol $c$, the following holds:
$$\forall 1\le i\le n. d_i(c(t_1,...,t_n))=t_i$$
Moreover, we usually axiomatise every term algebra with the \textit{injectivity}, \textit{distinctness}, \textit{exhaustiveness} and \textit{acyclicity} axioms. The inductive types we use in this paper are:
$$\begin{aligned}\nat&:=0 &&\mid \suc(\pre(\nat))\\
\lst&:=\nil&&\mid\cons(\head(\nat),\tail(\lst))\\
\btree&:=\bnil &&\mid \bnode(\bleft(\btree),\bval(\nat),\bright(\btree))\end{aligned}$$

An \textit{interpreted symbol} is a function or predicate whose meaning is defined through axioms, e.g. $=$ is an interpreted symbol in first-order logic with equality. All other symbols are called \textit{uninterpreted}. We distinguish \textit{function/predicate definitions} from regular axioms. These define a branch of computation for a function/predicate. Such axiom is denoted by marking exactly one equality literal in it with $:=$ such as $F\rightarrow l:=r$ which means that the orientation of this equality is fixed as left-to-right, $l$ is a function header and $r$ is a function definition, $F$ is the guard condition for this branch. We abuse this notation for predicate definitions where $:=$ can be replaced with a $\leftrightarrow$.

A notation we use is $E[s]$ meaning there is zero \todo{is this okay with zero? used in IndM rule} (or more) distinguished occurrence(s) of the term $s$ in $E$. $E[t]$ then means that these occurrences are changed to a term $t$. We may abbreviate $E[t_1]...[t_n]$ with $E[\overline{t}]$.

\subsection{Saturation-based proof search}
Given a set of input formulas $C$ in clausal form, the set of all derivable clauses using an inference system from the set $C$ is called the \textit{closure of $F$} w.r.t. the system. If the closure contains $\square$, the original set $C$ is unsatisfiable, otherwise it is satisfiable. The process of computing the closure is called \textit{saturation}. In practice more subtle notions are needed to tackle this problem, the first one is \textit{saturation up to redundancy}. A clause $C$ is \textit{redundant} w.r.t a set of clauses $S$ if some subset of $S$ of clauses smaller than $C$ w.r.t $\succ$ logically imply $C$. An inference system is usually equipped with \textit{simplification and deletion rules} to get rid of redundant clauses. Second, selection methods are used which control the order in which the inferences are applied. For a more detailed discussion on saturation algorithms see \cite{cav13}.

We use the \textit{superposition calculus} as the inference system in this paper. It works on sets of clauses -- we denote the conversion of a formula $F$ to its \textit{clausal normal form} with $\mathtt{cnf}(F)$. The superposition calculus is \textit{sound} and \textit{refutationally complete}. A \textit{refutation} is a derivation of $\bot$. Refutational completeness means for any unsatisfiable formula set, we can derive the empty clause. Therefore, with superposition we usually negate our input conjecture and try to refute it which, if successful, means the original conjecture is valid.

For the completeness a simplification term ordering $\succ$ of terms is needed (e.g. KBO, LPO) which is extended to literals and clauses with the multiset-extension. We will abuse notation and use the same symbol $\succ$ to denote the original ordering and its extensions.

We introduce an inference rule for induction adapted from \cite{vampireinduction}:
\begin{equation*}\label{eq:ind:rule}
\infer[\small{\tt (Ind)}]{\mathtt{cnf}(F \rightarrow \forall y.(L[y]))}{\lnot L[t] \lor C}
\end{equation*}
where $L$ is a ground literal, $C$ is a clause and $F \rightarrow \forall y.(L[y])$ is some valid induction formula. The conclusion of the rule can be then resolved against its
premise, yielding the clauses $\mathtt{cnf}(\lnot F) \lor C$.

	\section{Proving equational goals}
	\label{sec:equations}
	% !TeX root = ./paper.tex

\section{Proving equational goals}
In general, proving first-order theorems regarding inductive types requires not only using inference rules, but also coming up with the "right" induction formula when needed which is known to be undecidable.

Moreover, as suggested by Example \ref{ex:1}, induction and the superposition calculus seem to be incompatible since we have to sometimes go against the ordering to get a proof. This apparent incompatibility does not contradict any completeness results and comes rather from the automation of induction -- if we know \textit{a priori} what induction formulas to use, putting them into the search space and letting any standard superposition prover do its work will give a proof.

In this part, we present how special treatment of function definitions and induction hypotheses of equational induction formulas leads to more proofs.

\subsection{Function definitions}
In this paper, we consider well-founded and well-defined, possibly recursive but not mutually recursive function definitions whose branches are represented by one or more axioms in the search space. We assume that each of these axioms contains exactly one marked equality ($\rightsquigarrow$) with a fixed orientation:
$$C\lor \mathtt{f}(\overline{s}) \rightsquigarrow t$$
These axioms are only used as rewrite rules with the following two inference rules:
\begin{equation}
\infer[\small{(\texttt{DemF})}]{L[t\theta]\lor D}
{f(\overline{s})\rightsquigarrow t & \cancel{L[f(\overline{s})\theta]\lor D}}
\end{equation}
where $f(\overline{s})\theta\succ t\theta$ and $L[f(\overline{x})\theta]\lor D\succ f(\overline{x})\theta=t\theta$. Since for functions containing no mutually recursive functions, the function "dependency tree" is a DAG, it is expected that this rule can be used at least for the base cases of any function definition since those definitions can be oriented in the right way by simply creating a precedence from the DAG.

Second, all other non-simplifying rewriting with function definitions is done with the modified paramodulation rule:
\begin{equation}
\infer[\small{(\texttt{ParF})}]{L[t\theta]\lor C\theta\lor D}{f(\overline{s})\rightsquigarrow t\lor C & L[f(\overline{s})\theta]\lor D}
\end{equation}
This rule has no side conditions so that we can rewrite any part of a clause which helps expand function headers when needed but at the same time we lose properties such as fairness and completeness.

\subsection{Induction hypotheses}
We follow in general the induction principles used in \cite{vampireinduction,vampiregeneralization}, so our main induction rule is the following:
\begin{equation*}
\infer[\small{\tt (Ind)}]{\mathtt{cnf}(F \rightarrow \forall \overline{y}.L[\overline{y}])}{\overline{L}[\overline{t}] \lor C}
\end{equation*}
where $L$ is a ground literal, $C$ is a clause and $F \rightarrow \forall \overline{y}.L[\overline{y}]$ is a valid induction formula. $\overline{y}$ is a tuple of variables and $\overline{t}$ is a tuple of induction terms, both of size $m$.

In particular, from a case distinction $t_{11},...,t_{1{m_1}}\rightarrow t_1\mid...\mid t_{n1},...,t_{n{m_n}}\rightarrow t_n$, we create induction formulas of the form:
$$\bigwedge_{i=1}^{n}\forall \overline{z_i}.\bigg(\wedge_{j=1}^{m_i}L[\overline{t_{ij}}]\rightarrow L[\overline{t_i}]\bigg)\rightarrow \forall \overline{y}.L[\overline{y}]$$
Upon clausifying this formula, we create for each case $t_{i1},...,t_{i{m_i}}\rightarrow t_i$ a Skolemized case $t_{i1}^\prime,...,t_{i{m_i}}^\prime\rightarrow t_i^\prime$ where each variable in $t_i$ is replaced with a fresh Skolem constant. In the clausified form, we make distinction of induction hypotheses $L[t_{i{m_i}}^\prime]$ and their corresponding induction conclusions $\overline{L}[t_i^\prime]$. We propagate this information in each inference.

The induction hypothesis rewriting rule is then as follows:
\begin{equation*}
\infer[\small{\tt (IndHRW)}]{\mathtt{cnf}(F \rightarrow \forall \overline{y}.(s[r]=t)[\overline{y}])}{l=r \lor D& s[l]\neq t \lor C}
\end{equation*}
where $l=r$ is marked as induction hypothesis for $s\neq t$ and $F \rightarrow \forall \overline{y}.(s[r]=t)[\overline{y}]$ is a valid induction formula.

Note that the sides of $s\neq t$ are not distinguished, that is, we can perform the inference on either sides but not on both in the same inference. Since the goal of rewriting a conclusion with its hypothesis is to make the sides equal or at least introduce some common structure, it makes sense to rewrite only one side.

In fact, we can match the sides of an equality conclusion with that of its induction hypothesis. In Example \ref{ex:1}, $\rev(\rev(y))\sim\rev(\rev(\cons(z,y)))$ and $y\sim\cons(z,y)$, since they originate from the same sides of the inducted on literal.

\todo[inline]{Add example about multiple hypotheses used at once}


	\section{Proving non-equational goals}
	\label{sec:predicates}
	% !TeX root = ./paper.tex

\section{Proving non-equational goals}
\begin{equation*}\label{eq:multind}
\infer[\small{\tt (IndM)}]{\mathtt{cnf}(F \rightarrow \forall \overline{y}.(\bigwedge_{1\le i\le n}L_i[\overline{y}]\rightarrow L[\overline{y}]))}{L_1[\overline{t}]\lor C_1\quad...\quad L_n[\overline{t}]\lor C_n\quad\lnot L[\overline{t}] \lor C}
\end{equation*}
where $L$ and $L_i$, $1\le i\le n$ are ground literals, $C$ and $C_i$, $1\le i\le n$ are clauses and $F \rightarrow \forall \overline{y}.(\bigwedge_{1\le i\le n}L_i[\overline{y}]\rightarrow L[\overline{y}])$ is a valid induction formula. $\overline{y}$ is a tuple of variables and $\overline{t}$ is a tuple of induction terms, both of size $m$. We once again binary resolve each resulting clause with the original literals to get $\mathtt{cnf}(\lnot F) \lor\bigvee_{1\le i\le n} C_i\lor C$.


	\section{Induction formulas}
	\label{sec:induction_formulas}
	% !TeX root = ./paper.tex

There are many ways to generate induction formulas \cite{aclhandbook,bundychapter,cruanes,hipspec,hipster,isaplanner,computing}. In this section, we utilise function and predicate definitions to come up with them. A function or predicate has a number of branches, these are characterized by one or more clauses.  Two clauses $C\lor \mathtt{f}(\overline{s_1})\rightsquigarrow t_1$ and $D\lor \mathtt{f}(\overline{s_2})\rightsquigarrow t_2$ belong to the same branch of a function $\mathtt{f}$ if $\mathtt{f}(\overline{s_1})$ and $\mathtt{f}(\overline{s_2})$ are variants of each other. Similarly, two clauses $C\lor (\neg)\mathtt{P}(\overline{s_1})$ and $D\lor (\neg)\mathtt{P}(\overline{s_2})$ belong to the same branch of a predicate $\mathtt{P}$ if $\mathtt{P}(\overline{s_1})$ and $\mathtt{P}(\overline{s_2})$ are variants of each other. Therefore, we can characterize a branch with its characteristic term $\mathtt{f}(\overline{s})$ or atom $\mathtt{P}(\overline{t})$.

For a function, the recursive calls corresponding to one of its branches $\mathtt{f}(\overline{s})$ is the set:
$$\mathcal{R}(\mathtt{f}(\overline{s})):=\{\mathtt{f}(\overline{s^\prime})\mid \mathtt{f}(\overline{s^\prime})\trianglelefteq t, C\lor \mathtt{f}(\overline{s}) \rightsquigarrow t\}$$

For a predicate branch $\mathtt{P}(\overline{t})$, the set is:
$$\mathcal{R}(\mathtt{P}(\overline{t})):=\{\mathtt{P}(\overline{s^\prime})\in C, \mathtt{P}(\overline{t})
\lor C\}$$

\todo[inline]{Argue why we are not taking guard conditions into account when generating an induction formula. The reason is more or less that after rewriting a literal stemming from induction $L[f(c(\sigma))]\lor C$ with all corresponding function definitions $f(c(x))\lor C_1$, ..., $f(c(x))\lor C_n$, we get $n$ clauses which -- given the functions are well-defined -- can be binary resolved against each other in the end. Also, due to well-definedness, there cannot be any function definition that we can demodulate (or subsumption demodulate) with, so this always results in the $n$ branches being expanded (except when $C$ already contains the matching condition from one $C_i$ but then by well-definedness, the others would be unnecessary)}

We categorize argument positions $1\le i\le n$ based on \cite{cruanes}. An argument position for function $\mathtt{f}$ is \textit{active} if there is a branch characterized by $\mathtt{f}(\overline{s})$ s.t. $s_i \trianglerighteq s^\prime_i$ for some $\mathtt{f}(\overline{s^\prime})\in\mathcal{R}(\mathtt{f}(\overline{s}))$. An argument position is \textit{accumulator} if $s_i$ is a variable and $s_i\neq s^\prime_i$. We denote the active argument positions of $\mathtt{f}$ with $I_\mathtt{f}$.

Given a function term $\mathtt{f}(\overline{c})$, we define a case distinction over the terms $\{c_i\mid i\in I_\mathtt{f}\}$ as follows: we create a new term $\mathtt{f}(\overline{c^\prime})$ where we replace each $c_i$ with $i\in I_\mathtt{f}$ with fresh variables. Let $\theta$ be the mgu for each branch characterized by $\mathtt{f}(\overline{s})$ and $\mathtt{f}(\overline{c})$ and let $\mathtt{f}(\overline{t}):=\mathtt{f}(\overline{s})\theta=\mathtt{f}(\overline{c^\prime})\theta$. If the two terms can be unified, we create a mapping $\{c_i\mapsto t_i\mid i\in I_\mathtt{f}\}$. This tells what terms to substitute for the induction terms in this case of the induction formula. Then, we create the mappings for the induction hypotheses corresponding to this case: we take each $\mathtt{f}(\overline{s^\prime})\theta\in\mathcal{R}(\mathtt{f}(\overline{s}))$, and unify them with $\mathtt{f}(\overline{c^\prime})$. The unifications, if they exist, give rise to similar mappings.

\begin{example}\label{ex:3}
	Given the function definition $\qreva : (\lst, \lst)\rightarrow\lst$:
	\begin{align}
	&\qreva(\nil)\rightsquigarrow y\label{qreva1}\tag{qreva.1}\\ &\qreva(\cons(u,v))\rightsquigarrow\qreva(v,\cons(u,y))\label{qreva2}\tag{qreva.2}\end{align}
	And the unit clause:
	$$\qreva(\sigma_0,\sigma_1)\neq\append(\rev(\sigma_0),\sigma_1)$$
	Let us first look at the function term $\qreva(\sigma_0,\sigma_1)$:
	We create the term with variables $\qreva(x,\sigma_1)$. Then, we unify with each
	header and recursive call of $\qreva$:
	\begin{enumerate}
		\item $\qreva(x,\sigma_1)$ unified with $\qreva(\nil,y)$ gives the mapping $\{\sigma_0\mapsto\nil\}$.
		\item $\qreva(x,\sigma_1)$ unified with $\qreva(\cons(u,v),y)$ gives $\{\sigma_0\mapsto\cons(u,v)\}$
		\item To obtain the hypothesis of the second case, we first apply the unifier on its recursive call: $\qreva(v, \cons(u,\sigma_1))$. The mapping for the corresponding hypothesis is then $\{\sigma_0\mapsto v\}$.
	\end{enumerate}
	We use the mappings to create an induction formula: each mapping defines what to replace in the corresponding cases and induction hypotheses. Moreover, we add an induction conclusion matching the induction terms:
	\begin{align}
    \nonumber	&\Big(\qreva(\nil,\sigma_1)=\append(\rev(\nil),\sigma_1)\land\\
	\begin{split}\nonumber&\forall u,v.\big(\qreva(v,\sigma_1)=\append(\rev(v),\sigma_1)\rightarrow\\
	&\qreva(\cons(u,v),\sigma_1)=\append(\rev(\cons(u,v)),\sigma_1)\big)\Big)\end{split}\\
    \nonumber&\rightarrow \forall x.\qreva(x,\sigma_1)=\append(\rev(x),\sigma_1)
	\end{align}
\end{example}

The above method -- when used on each predicate and function term in a literal -- gives "matching" induction formulas in a sense that in each case of an induction formula by definition at least the term this formula was generated from will be an instance of the corresponding function/predicate definition.

The method can be improved in two regards: first, we can recursively mark terms in a literal in active positions and only generate formulas from those, since all others will get stuck at some point. Second, we can create terms to generate from based on subterms in active arguments.

For example the following literal contains two terms that can generate induction formulas:
$$\neg\fleq(\sigma_0,\add(\sigma_1,\sigma_0))$$
The first one gives induction terms $\sigma_0$ and $\add(\sigma_1,\sigma_0)$, the second $\sigma_1$, whereas we need one where we induct on $\sigma_0$ and $\sigma_1$ simultaneously, to match the recursive case of $\fleq$ after simplifying with $\add$. But when looking at the $\fleq$ term, instead of only taking the $\add$ term as the second argument, we look inside its active arguments and take $\sigma_1$ also (essentially generating from term $\fleq(\sigma_0, \sigma_1)$).

More generally, when considering a term $\mathtt{f}(t_1,...,t_n)$, we take all subterms in its active argument positions $t_i$ with the same sort as $t_i$, and take all combinations (leaving non-active arguments).
\subsection{Strengthening the hypotheses}
In Example \ref{ex:3}, after clausification and rewriting with function definitions, the base case is solved, but we are left with the following two unit clauses (the step case conclusion and hypothesis):
$$\qreva(\sigma_2,\sigma_1)=\append(\rev(\sigma_2),\sigma_1)$$
$$\qreva(\sigma_2,\cons(\sigma_3,\sigma_1))\neq\append(\append(\rev(\sigma_2),\cons(\sigma_3,\nil)),\sigma_1)$$
An easy solution for this is to use a \textit{strengthened hypothesis}, where we replace $\sigma_1$ with a universally quantified variable, so that the left-hand sides will match:
$$\qreva(\sigma_2,z)=\append(\rev(\sigma_2),z)$$
Yet, this "too strong" hypothesis can backfire in many ways: not only does it burden the solver by more potential rewrites but the prover may even demodulate the wrong side of the step conclusion after which the proof is ultimately stuck:
$$\qreva(\sigma_2,\cons(\sigma_3,\sigma_1))\neq\append(\qreva(\sigma_2,\cons(\sigma_3,\nil),\sigma_1))$$
Our proposed solution is to use accumulator argument positions to gives us hints at possible non-trivial hypotheses. In Example \ref{ex:3}, unifying with the recursive call gives $\qreva(v, \cons(u,\sigma_1))$, which means we can replace $\sigma_1$ -- after Skolemizing $\cons(u,\sigma_1)$ -- with $\cons(\sigma_3,\sigma_1)$:
$$\qreva(\sigma_2,\cons(\sigma_3,\sigma_1))=\append(\rev(\sigma_2),\cons(\sigma_3,\sigma_1))$$
The left-hand sides now match exactly, so applying the hypothesis gives:
$$\append(\rev(\sigma_2),\cons(\sigma_3,\sigma_1))\neq\append(\append(\rev(\sigma_2),\cons(\sigma_3,\nil)),\sigma_1)$$

\subsection{Generalizations}
As described in \cite{vampiregeneralization}, inducting on all occurrences of an induction term may lead to a failed proof even if the case distinction was the right one. Inspired by \cite{cruanes}, we devised a heuristic that takes \textit{all active occurrences} of an induction term. So for example, we generalize the literal $\forall x.\fleq(x,\add(x,x))$ to $\forall x,z.\fleq(z,\add(z,x))$.

But sometimes we have to take all occurrences, e.g. in $\forall x.\double(x) = \add(x,x)$, we have to take the last, non-active one as well. Guessing which generalization works takes up a lot of resources, therefore, we instead add generalized induction formulas to the search space for both only active and all occurrences variants. So we generalize to $\forall x,z.\double(z) = \add(z,x)$ and $\forall z.\double(z) = \add(z,z)$ as well.


	
	\section{Experiments}
	\label{sec:experiments}
	% !TeX root = ./paper.tex
We implemented the discussed techniques in \textsc{Vampire} and experimented them on a large number of benchmarks. In particular, we used a generated set of 3000+ problems of increasing difficulty which we call the Vampire Induction Benchmarks (VIB for short). This explicitly contains the function definition orientations via the $\mathtt{define\mhyphen fun}$ and $\mathtt{define\mhyphen fun\mhyphen rec}$ keywords of SMT-LIB thus eliminating the possibility of errors that may arise from ill-oriented function definitions. We also looked at the UFDTLIA set of SMT-COMP that contains some well-known hard problems of inductive reasoning which required function definition discovery. \todo{Explain somewhere function definition discovery or leave it out completely.}

The experiments were conducted using \textsc{BenchExec} \cite{benchmarking,competitionresults} and each solver instance was provided one processor core, 16 GB of memory and a time limit of 300 seconds. Due to the large number of introduced flags, instead of showing individual results for each discussed technique, we present the results obtained by creating a custom schedule in \textit{portfolio mode} with all combinations of flags (a total of 64 combinations). The results are shown in Figure \ref{fig:comparison3}. \textsc{Vampire} refers to the original \textsc{Vampire}, $\textsc{Vampire}^{\ast\ast}$ to the new \textsc{Vampire} with only the new induction variant enabled and $\textsc{Vampire}^{\dag}$ to $\textsc{Vampire}^{\ast\ast}$ with the portfolio mode schedule. We measured two solvers ZipperPosition and CVC4, which showed the best results among all solvers for the VIB and UFDTLIA set, respectively.

\begin{figure}
	\centering
	\begin{tabular}[c]{c|c|c|c|c|c|c|c}
		\multicolumn{3}{c}{ } & \rot{30}{\textsc{Vampire}} & \rot{30}{$\textsc{Vampire}^{\ast\ast}$} & \rot{30}{$\textsc{Vampire}^{\dag}$} & \rot{30}{\textsc{CVC4}} & \rot{30}{\textsc{ZipperPosition}} \\ \hline
		\multicolumn{2}{c|}{Problems} & Count & \multicolumn{5}{c}{Solved problems} \\ \hline
		\multirow{3}{*}{$\lst$} & $\append$ & 313 & \cellcolor{blue!11!white}72 & \cellcolor{blue!38!white}238 & \cellcolor{blue!50!white}313 & \cellcolor{blue!0!white}1 & \cellcolor{blue!49!white}312\\
		& \texttt{pref} & 696 & \cellcolor{blue!6!white}94 & \cellcolor{blue!49!white}687 & \cellcolor{blue!50!white}696 & \cellcolor{blue!0!white}1 & \cellcolor{blue!41!white}583\\
		& $\rev$ & 4 & \cellcolor{blue!25!white}2 & \cellcolor{blue!25!white}2 & \cellcolor{blue!37!white}3 & \cellcolor{blue!0!white}0 & \cellcolor{blue!25!white}2\\
		\hline
		\multirow{9}{*}{$\nat$} & + assoc & 312 & \cellcolor{blue!12!white}81 & \cellcolor{blue!41!white}261 & \cellcolor{blue!50!white}312 & \cellcolor{blue!0!white}1 & \cellcolor{blue!50!white}312\\
		& + mix & 830 & \cellcolor{blue!6!white}103 & \cellcolor{blue!15!white}263 & \cellcolor{blue!35!white}591 & \cellcolor{blue!4!white}70 & \cellcolor{blue!33!white}552\\
		& + comm & 2 & \cellcolor{blue!50!white}2 & \cellcolor{blue!50!white}2 & \cellcolor{blue!50!white}2 & \cellcolor{blue!25!white}1 & \cellcolor{blue!50!white}2\\
		& $+ \suc 0$ mix & 486 & \cellcolor{blue!8!white}81 & \cellcolor{blue!12!white}124 & \cellcolor{blue!28!white}281 & \cellcolor{blue!8!white}87 & \cellcolor{blue!19!white}186\\
		& $\cdot$ & 6 & \cellcolor{blue!0!white}0 & \cellcolor{blue!0!white}0 & \cellcolor{blue!16!white}2 & \cellcolor{blue!0!white}0 & \cellcolor{blue!0!white}0\\
		& $\even$ & 2 & \cellcolor{blue!0!white}0 & \cellcolor{blue!0!white}0 & \cellcolor{blue!25!white}1 & \cellcolor{blue!0!white}0 & \cellcolor{blue!0!white}0\\
		& $\simeq$ & 4 & \cellcolor{blue!25!white}2 & \cellcolor{blue!37!white}3 & \cellcolor{blue!37!white}3 & \cellcolor{blue!25!white}2 & \cellcolor{blue!50!white}4\\
		& $\leq$ & 696 & \cellcolor{blue!8!white}117 & \cellcolor{blue!50!white}696 & \cellcolor{blue!50!white}696 & \cellcolor{blue!0!white}1 & \cellcolor{blue!41!white}583\\
		& $\dup$ & 1 & \cellcolor{blue!0!white}0 & \cellcolor{blue!0!white}0 & \cellcolor{blue!0!white}0 & \cellcolor{blue!0!white}0 & \cellcolor{blue!0!white}0\\
		\hline
		\multirow{1}{*}{$\btree$} & $\fltn$ & 3 & \cellcolor{blue!0!white}0 & \cellcolor{blue!0!white}0 & \cellcolor{blue!33!white}2 & \cellcolor{blue!16!white}1 & \cellcolor{blue!33!white}2\\
		\hline
		\multirow{1}{*}{$\nat\&\lst$} & $+\&\append$ & 1 & \cellcolor{blue!50!white}1 & \cellcolor{blue!50!white}1 & \cellcolor{blue!50!white}1 & \cellcolor{blue!0!white}0 & \cellcolor{blue!0!white}0\\
		\hline
		\multicolumn{2}{c|}{UFDTLIA} & 327 & \cellcolor{blue!22!white}144 & \cellcolor{blue!22!white}149 & \cellcolor{blue!26!white}176 & \cellcolor{blue!32!white}214 & -\\
		\hhline{|=|=|=|=|=|=|=|=|}
		\multicolumn{2}{c|}{All} & 3683 & \cellcolor{blue!9!white}699 & \cellcolor{blue!33!white}2426 & \cellcolor{blue!42!white}3079 & \cellcolor{blue!5!white}379 & \cellcolor{blue!38!white}2538\\
		\hline
		\multicolumn{8}{c}{ }\\
		\hline
		\multicolumn{2}{c|}{Problems} & Count & \multicolumn{5}{c}{Uniquely solved problems} \\ \hline
		\multirow{3}{*}{$\lst$} & $\append$ & 313 & 0 & 0 & \cellcolor{gray!20!white}1 & 0 & 0\\
		& $\pref$ & 696 & 0 & 0 & 0 & 0 & 0\\
		& $\rev$ & 4 & 0 & 0 & \cellcolor{gray!20!white}1 & 0 & 0\\
		\hline
		\multirow{9}{*}{$\nat$} & + assoc & 312 & 0 & 0 & 0 & 0 & 0\\
		& + mix & 830 & \cellcolor{gray!20!white}2 & 0 & \cellcolor{gray!20!white}116 & 0 & \cellcolor{gray!20!white}93\\
		& + comm & 2 & 0 & 0 & 0 & 0 & 0\\
		& $+ \suc 0$ mix & 486 & 0 & 0 & \cellcolor{gray!20!white}95 & 0 & \cellcolor{gray!20!white}3\\
		& $\cdot$ & 6 & 0 & 0 & \cellcolor{gray!20!white}2 & 0 & 0\\
		& $\even$ & 2 & 0 & 0 & \cellcolor{gray!20!white}1 & 0 & 0\\
		& $\simeq$ & 4 & 0 & 0 & 0 & 0 & 0\\
		& $\leq$ & 696 & 0 & 0 & 0 & 0 & 0\\
		& $\dup$ & 1 & 0 & 0 & 0 & 0 & 0\\
		\hline
		\multirow{1}{*}{$\btree$} & $\fltn$ & 3 & 0 & 0 & 0 & 0 & 0\\
		\hline
		\multirow{1}{*}{$\nat\&\lst$} & $+\&\append$ & 1 & 0 & 0 & 0 & 0 & 0\\
		\hline
		\multicolumn{2}{c|}{UFDTLIA} & 327 & \cellcolor{gray!20!white}2 & 0 & \cellcolor{gray!20!white}9 & \cellcolor{gray!20!white}50 & -\\
		\hhline{|=|=|=|=|=|=|=|=|}
		\multicolumn{2}{c|}{All} & 3683 & \cellcolor{gray!20!white}4 & 0 & \cellcolor{gray!20!white}225 & \cellcolor{gray!20!white}50 & \cellcolor{gray!20!white}96\\
		\hline
	\end{tabular}
	\caption{\textsc{Vampire} with and without portfolio mode and other solvers compared}
	\label{fig:comparison3}
\end{figure}


	\section{Related work}
	\label{sec:related_work}
	% !TeX root = ./paper.tex



	\section{Conclusion}
	The conclusion goes here.
	
	
	
	
	% conference papers do not normally have an appendix
	
	
	% use section* for acknowledgment
	\section*{Acknowledgment}
	
	
	The authors would like to thank...
	
	
	
	
	
	% trigger a \newpage just before the given reference
	% number - used to balance the columns on the last page
	% adjust value as needed - may need to be readjusted if
	% the document is modified later
	%\IEEEtriggeratref{8}
	% The "triggered" command can be changed if desired:
	%\IEEEtriggercmd{\enlargethispage{-5in}}
	
	% references section
	
	% can use a bibliography generated by BibTeX as a .bbl file
	% BibTeX documentation can be easily obtained at:
	% http://mirror.ctan.org/biblio/bibtex/contrib/doc/
	% The IEEEtran BibTeX style support page is at:
	% http://www.michaelshell.org/tex/ieeetran/bibtex/
	%\bibliographystyle{IEEEtran}
	% argument is your BibTeX string definitions and bibliography database(s)
	%\bibliography{IEEEabrv,../bib/paper}
	%
	% <OR> manually copy in the resultant .bbl file
	% set second argument of \begin to the number of references
	% (used to reserve space for the reference number labels box)
	

	\bibliographystyle{IEEEtran}

	\bibliography{paper-bib}
	
%	\begin{thebibliography}{1}
%		
%		\bibitem{IEEEhowto:kopka}
%		H.~Kopka and P.~W. Daly, \emph{A Guide to \LaTeX}, 3rd~ed.\hskip 1em plus
%		0.5em minus 0.4em\relax Harlow, England: Addison-Wesley, 1999.
%		
%	\end{thebibliography}
	
	
	
	
	% that's all folks
\end{document}


